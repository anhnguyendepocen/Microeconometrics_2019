\documentclass[]{book}
\usepackage{lmodern}
\usepackage{amssymb,amsmath}
\usepackage{ifxetex,ifluatex}
\usepackage{fixltx2e} % provides \textsubscript
\ifnum 0\ifxetex 1\fi\ifluatex 1\fi=0 % if pdftex
  \usepackage[T1]{fontenc}
  \usepackage[utf8]{inputenc}
\else % if luatex or xelatex
  \ifxetex
    \usepackage{mathspec}
  \else
    \usepackage{fontspec}
  \fi
  \defaultfontfeatures{Ligatures=TeX,Scale=MatchLowercase}
\fi
% use upquote if available, for straight quotes in verbatim environments
\IfFileExists{upquote.sty}{\usepackage{upquote}}{}
% use microtype if available
\IfFileExists{microtype.sty}{%
\usepackage{microtype}
\UseMicrotypeSet[protrusion]{basicmath} % disable protrusion for tt fonts
}{}
\usepackage{hyperref}
\hypersetup{unicode=true,
            pdftitle={Microeconometrics - e-Notes: Practice guide using R},
            pdfauthor={Jaime MONTANA},
            pdfborder={0 0 0},
            breaklinks=true}
\urlstyle{same}  % don't use monospace font for urls
\usepackage{natbib}
\bibliographystyle{apalike}
\usepackage{color}
\usepackage{fancyvrb}
\newcommand{\VerbBar}{|}
\newcommand{\VERB}{\Verb[commandchars=\\\{\}]}
\DefineVerbatimEnvironment{Highlighting}{Verbatim}{commandchars=\\\{\}}
% Add ',fontsize=\small' for more characters per line
\usepackage{framed}
\definecolor{shadecolor}{RGB}{248,248,248}
\newenvironment{Shaded}{\begin{snugshade}}{\end{snugshade}}
\newcommand{\KeywordTok}[1]{\textcolor[rgb]{0.13,0.29,0.53}{\textbf{#1}}}
\newcommand{\DataTypeTok}[1]{\textcolor[rgb]{0.13,0.29,0.53}{#1}}
\newcommand{\DecValTok}[1]{\textcolor[rgb]{0.00,0.00,0.81}{#1}}
\newcommand{\BaseNTok}[1]{\textcolor[rgb]{0.00,0.00,0.81}{#1}}
\newcommand{\FloatTok}[1]{\textcolor[rgb]{0.00,0.00,0.81}{#1}}
\newcommand{\ConstantTok}[1]{\textcolor[rgb]{0.00,0.00,0.00}{#1}}
\newcommand{\CharTok}[1]{\textcolor[rgb]{0.31,0.60,0.02}{#1}}
\newcommand{\SpecialCharTok}[1]{\textcolor[rgb]{0.00,0.00,0.00}{#1}}
\newcommand{\StringTok}[1]{\textcolor[rgb]{0.31,0.60,0.02}{#1}}
\newcommand{\VerbatimStringTok}[1]{\textcolor[rgb]{0.31,0.60,0.02}{#1}}
\newcommand{\SpecialStringTok}[1]{\textcolor[rgb]{0.31,0.60,0.02}{#1}}
\newcommand{\ImportTok}[1]{#1}
\newcommand{\CommentTok}[1]{\textcolor[rgb]{0.56,0.35,0.01}{\textit{#1}}}
\newcommand{\DocumentationTok}[1]{\textcolor[rgb]{0.56,0.35,0.01}{\textbf{\textit{#1}}}}
\newcommand{\AnnotationTok}[1]{\textcolor[rgb]{0.56,0.35,0.01}{\textbf{\textit{#1}}}}
\newcommand{\CommentVarTok}[1]{\textcolor[rgb]{0.56,0.35,0.01}{\textbf{\textit{#1}}}}
\newcommand{\OtherTok}[1]{\textcolor[rgb]{0.56,0.35,0.01}{#1}}
\newcommand{\FunctionTok}[1]{\textcolor[rgb]{0.00,0.00,0.00}{#1}}
\newcommand{\VariableTok}[1]{\textcolor[rgb]{0.00,0.00,0.00}{#1}}
\newcommand{\ControlFlowTok}[1]{\textcolor[rgb]{0.13,0.29,0.53}{\textbf{#1}}}
\newcommand{\OperatorTok}[1]{\textcolor[rgb]{0.81,0.36,0.00}{\textbf{#1}}}
\newcommand{\BuiltInTok}[1]{#1}
\newcommand{\ExtensionTok}[1]{#1}
\newcommand{\PreprocessorTok}[1]{\textcolor[rgb]{0.56,0.35,0.01}{\textit{#1}}}
\newcommand{\AttributeTok}[1]{\textcolor[rgb]{0.77,0.63,0.00}{#1}}
\newcommand{\RegionMarkerTok}[1]{#1}
\newcommand{\InformationTok}[1]{\textcolor[rgb]{0.56,0.35,0.01}{\textbf{\textit{#1}}}}
\newcommand{\WarningTok}[1]{\textcolor[rgb]{0.56,0.35,0.01}{\textbf{\textit{#1}}}}
\newcommand{\AlertTok}[1]{\textcolor[rgb]{0.94,0.16,0.16}{#1}}
\newcommand{\ErrorTok}[1]{\textcolor[rgb]{0.64,0.00,0.00}{\textbf{#1}}}
\newcommand{\NormalTok}[1]{#1}
\usepackage{longtable,booktabs}
\usepackage{graphicx,grffile}
\makeatletter
\def\maxwidth{\ifdim\Gin@nat@width>\linewidth\linewidth\else\Gin@nat@width\fi}
\def\maxheight{\ifdim\Gin@nat@height>\textheight\textheight\else\Gin@nat@height\fi}
\makeatother
% Scale images if necessary, so that they will not overflow the page
% margins by default, and it is still possible to overwrite the defaults
% using explicit options in \includegraphics[width, height, ...]{}
\setkeys{Gin}{width=\maxwidth,height=\maxheight,keepaspectratio}
\IfFileExists{parskip.sty}{%
\usepackage{parskip}
}{% else
\setlength{\parindent}{0pt}
\setlength{\parskip}{6pt plus 2pt minus 1pt}
}
\setlength{\emergencystretch}{3em}  % prevent overfull lines
\providecommand{\tightlist}{%
  \setlength{\itemsep}{0pt}\setlength{\parskip}{0pt}}
\setcounter{secnumdepth}{5}
% Redefines (sub)paragraphs to behave more like sections
\ifx\paragraph\undefined\else
\let\oldparagraph\paragraph
\renewcommand{\paragraph}[1]{\oldparagraph{#1}\mbox{}}
\fi
\ifx\subparagraph\undefined\else
\let\oldsubparagraph\subparagraph
\renewcommand{\subparagraph}[1]{\oldsubparagraph{#1}\mbox{}}
\fi

%%% Use protect on footnotes to avoid problems with footnotes in titles
\let\rmarkdownfootnote\footnote%
\def\footnote{\protect\rmarkdownfootnote}

%%% Change title format to be more compact
\usepackage{titling}

% Create subtitle command for use in maketitle
\providecommand{\subtitle}[1]{
  \posttitle{
    \begin{center}\large#1\end{center}
    }
}

\setlength{\droptitle}{-2em}

  \title{Microeconometrics - e-Notes: Practice guide using R}
    \pretitle{\vspace{\droptitle}\centering\huge}
  \posttitle{\par}
    \author{Jaime MONTANA}
    \preauthor{\centering\large\emph}
  \postauthor{\par}
      \predate{\centering\large\emph}
  \postdate{\par}
    \date{Updated the 2019-10-07}

\usepackage{booktabs}

\begin{document}
\maketitle

{
\setcounter{tocdepth}{1}
\tableofcontents
}
\chapter{Introduction}\label{introduction}

\begin{center}\includegraphics[width=0.9\linewidth]{./images/1200px-Logo_pse_petit} \end{center}

\section{General Info}\label{general-info}

\subsection{Contact information}\label{contact-information}

\begin{itemize}
\tightlist
\item
  \textbf{Webpage:} {[}Under construction{]}
\item
  \textbf{E-mail:}
  \href{mailto:jaimem.montana@gmail.com}{\nolinkurl{jaimem.montana@gmail.com}}
\item
  \textbf{Slack:}
  \href{https://join.slack.com/t/psemicecometrics19/shared_invite/enQtNzY4NTYyMTM2MTMzLTM5ZmVmNTRjYTg3NzVkZTg4YzQzNjU3OTllZTFkODJjYzA0MWE0MTZkOTFlZmZhODhiMmViZWNlYzRiZWI1ZjU}{Follow
  this link}
\end{itemize}

Feel free to email at any time to ask questions about the methods
covered in class, although I will prioritize communications over the
Slack channel. In this way everyone can benefit from others' questions
and answers. If anyone knows how to solve a problem, debug or fix the
code in the Slack forum s/he can help.

There are some rules for making questions on the procedures. Before
asking, it is \textbf{mandatory} that you consult the documentation of
the function/package; also try to search the answer in a public forum
(i.e.~Stack Overflow). If after that you still have troubles, post the
question taking the following into consideration:

\begin{itemize}
\tightlist
\item
  Be clear and concise, so everyone can understand you
\item
  Be as specific as possible, being clear and straightforward
\item
  Include sufficient information: your goal, the code, the data,
  everething in order to reproduce the error
\end{itemize}

Also, you can ask questions about the interpretation of the results, the
theory behind, and the like.

\section{Objective}\label{objective}

This class notes are an interactive e-material for the Microeconometrics
course in the master APE in Paris School of Economics. The aim of this
notes is to provide an e-learning material to apply the theorical
concepts of the class. The notes are in \textbf{open review}: comments,
corrections or contributions that you can make to this part of the
course and the material provided are more than welcome.

This part of the course does not cover the theory, and assumes you
already had it covered and understood. We will depart from the theory
with direct application of the methods.

\section{Prerequisites}\label{prerequisites}

Please install R and Rstudio in your laptop. Here is a video to install
R and Rstudio in windows and mac. If you have questions or you could not
manage to install it, bring your laptop next session. I will help out
for the installation.

\begin{itemize}
\tightlist
\item
  windows-os: \href{https://www.youtube.com/watch?v=9-RrkJQQYqY}{For
  Windows click this link}
\item
  mac-os: \href{https://www.youtube.com/watch?v=GLLZhc_5enQ}{For Mac-OS
  click this link}
\end{itemize}

Why \textbf{R}? R is a \textbf{free} software programming language and a
software environment for statistical computing and graphics. The R
language is widely used among statisticians, economist, in finance and
academics circles.

\begin{itemize}
\item
  R is a \textbf{free} software, easy to install and runs in multiple
  OS.
\item
  A lot of documentation and forums. Excellent documentation on
  packages.
\item
  Very active community which allow to use other people codes and
  projects.
\item
  \textbf{Great} visualization tools thanks to \emph{ggplot} and
  \emph{plotly} packages.
\item
  If you understand the logic behind R you will get into every
  statistical software very easily.
\item
  Everything seems hard at the beginning. Just try and ask.
\end{itemize}

\textbf{A prior knowledge on the use of R is required.} We don`t have
much time to cover the basics. For an introduction to R you can check
the following material:

\begin{itemize}
\tightlist
\item
  \href{https://www.econometrics-with-r.org/1-1-a-very-short-introduction-to-r-and-rstudio.html}{Introduction
  to Econometrics with R, Chapter 1 - 6, by Christoph Hanck et al.}
\item
  \href{https://scpoecon.github.io/ScPoEconometrics/R-intro.html}{Introduction
  to Econometrics with R by Florian Oswald, Jean-Marc Robin and Vincent
  Viers}
\end{itemize}

\section{Course structure}\label{course-structure}

\begin{itemize}
\item
  We will have only 3 sessions (2 hours each)
\item
  Bring your laptop with R installed on it
\item
  The material for each session will be online just before each session.
  In that way you can follow from the e-notes and do the exercises with
  me during class
\item
  What to expect from each session:

\begin{verbatim}
1. Brief explanation on the method (how it works)
2. Replication of a published paper that applies the method (downloading data, cleaning data, estimation, tables and plotting results)
3. Discussion on the interpretation of the results
4. Q/A
\end{verbatim}
\item
  There will be \emph{suggested exercises}. These are \textbf{not
  mandatory}, but remember that if you want to master something, you
  need to practice. I will be happy to give some feedback on the
  suggested exercises if you want.
\end{itemize}

\chapter{Session I - Quantile
regression}\label{session-i---quantile-regression}

\begin{center}\includegraphics[width=0.9\linewidth]{./images/1200px-Logo_pse_petit} \end{center}

\section{Objective}\label{objective-1}

This first class is to introduce your to using R for implementing
quantile regressions. Therefore, we are going to:

\begin{enumerate}
\def\labelenumi{\arabic{enumi}.}
\item
  Understand the mathematical procedure behind the QR estmation and its
  computation (both for the estimates and for the the standard errors).
  This will help understanding the interpretation of the results
  obtained when applying the procedure in other contexts.
\item
  Reproduce a paper's tables and results using quantile regression

\begin{verbatim}
a) Import the data
b) Clean the data
c) Reproduce the summary statistics table
d) reproduce the regressions tables
\end{verbatim}
\item
  Interpret the results
\item
  Ways to communicate the results (plotting in R)
\end{enumerate}

In the last part of the lecture I will just mention and make reference
to other classes of QR estimators so you can investigate more on them;

\section{Quantile Regression}\label{quantile-regression}

\textbf{For a summary on what is the intuition and objective of quantile
regression check the article ``Quantile Regression''
\citep{koenker2001quantile}.}

QR is a method that allows you to analyse the relation between \(x\) and
\(y\) across the \(y\) distribution. It is useful when the researcher
thinks there are \emph{heterogeneous effects} at different values of the
indipendent variable. It is important to remark that the heterogeneity
is on the \textbf{outcome} \(y\). Also, it is widely used in presence of
outliers and extreme events (infinite variance), for OLS is inconsistent
in such cases while the median is always defined and consistent. For
quantiles other than \(\tau = 0.5\) the estimation is robust, too.

From the class we know the relationship between the \textbf{definition
of the estimator}, the \textbf{risk function} used in the optimization
(in the case of the lector the \emph{`LAD function'}, when
\(\tau = 0.5\)) and that we need to solve numerically the
\textbf{optimization program} in order to identify the parameters of
interest. Accordingly, we will explain how the algorithm works and we
are going to perform the numerical optimization by hand from the
simplest case to more complex problems.

\subsection{Geometric interpretation}\label{geometric-interpretation}

From \citep{koenker2001quantile}:

\begin{quote}
Quantiles seem inseparably linked to the operations of ordering and
sorting the sample observations that are usually used to define them. So
it comes as a mild surprise to observe that we can define \textbf{the
quantiles} through a simple alternative expedient \textbf{as an
optimization problem}. Just as we can define the sample mean as the
solution to the problem of minimizing a sum of squared residuals, we can
define the \emph{median as the solution to the problem of minimizing a
sum of absolute residuals}. The symmetry of the piecewise linear
absolute value function implies that the minimization of the sum of
absolute residuals must equate the number of positive and negative
residuals, thus assuring that there are the same number of observations
above and below the median. What about the other quantiles? Since the
symmetry of the absolute value yields the median, perhaps minimizing a
sum of asymmetrically weighted absolute residuals---simply giving
differing weights to positive and negative residuals---would yield the
quantiles. This is indeed the case.
\end{quote}

The slope of the coefficient is dividing the error space in two parts
according to the desired proportion. It is important to notice that we
are considering the error space, we are referring to the conditioning
quantile. The difference with the OLS is then clear since the two
processes are not comparable. While OLS might provide causal linkages,
this is prevented in QR precisely for this reason.

Let's generate some data to see how the line bisects the error space.
Since we are generating random data the first thing is to set a seed so
our example is reproducible. Then, we generate variance for our error
term (not constant) and set an intercept and define the slope. We set
everything in a `data.frame' objecto to plot it using the package
`ggplot2'\citep{R-ggplot2}.

\begin{Shaded}
\begin{Highlighting}[]
\KeywordTok{set.seed}\NormalTok{(}\DecValTok{464}\NormalTok{)}
\NormalTok{npoints <-}\StringTok{ }\DecValTok{150}
\NormalTok{x <-}\StringTok{ }\KeywordTok{seq}\NormalTok{(}\DecValTok{0}\NormalTok{,}\DecValTok{10}\NormalTok{,}\DataTypeTok{length.out =}\NormalTok{ npoints)        }
\NormalTok{sigma <-}\StringTok{ }\FloatTok{0.1} \OperatorTok{+}\StringTok{ }\FloatTok{0.25}\OperatorTok{*}\KeywordTok{sqrt}\NormalTok{(x) }\OperatorTok{+}\StringTok{ }\KeywordTok{ifelse}\NormalTok{(x}\OperatorTok{>}\DecValTok{6}\NormalTok{,}\DecValTok{1}\NormalTok{,}\DecValTok{0}\NormalTok{)                    }
\NormalTok{intercept <-}\StringTok{ }\DecValTok{2}                                
\NormalTok{slope <-}\StringTok{ }\FloatTok{0.2}                              
                             
\NormalTok{error <-}\StringTok{ }\KeywordTok{rnorm}\NormalTok{(npoints,}\DataTypeTok{mean =} \DecValTok{0}\NormalTok{, }\DataTypeTok{sd =}\NormalTok{ sigma)      }
\NormalTok{y <-}\StringTok{ }\NormalTok{intercept }\OperatorTok{+}\StringTok{ }\NormalTok{slope}\OperatorTok{*}\NormalTok{x }\OperatorTok{+}\StringTok{ }\NormalTok{error                    }
\NormalTok{dat <-}\StringTok{ }\KeywordTok{data.frame}\NormalTok{(x,y)}
\end{Highlighting}
\end{Shaded}

Let's plot our synthetic data. We are going to plot the line crossing
the 90th percentile conditioning on \(X\) (dashed red line), the OLS
curve (blue line with confidence intervals in grey) and the LAD
regression (regression to the median, \(\tau = 0.5\))

\begin{Shaded}
\begin{Highlighting}[]
\KeywordTok{ggplot}\NormalTok{(dat, }\KeywordTok{aes}\NormalTok{(x,y)) }\OperatorTok{+}\StringTok{ }\KeywordTok{geom_point}\NormalTok{() }\OperatorTok{+}\StringTok{ }\KeywordTok{geom_smooth}\NormalTok{(}\DataTypeTok{method=}\StringTok{"lm"}\NormalTok{) }\OperatorTok{+}\StringTok{ }
\StringTok{        }\KeywordTok{geom_quantile}\NormalTok{(}\DataTypeTok{quantiles =} \FloatTok{0.9}\NormalTok{, }\DataTypeTok{colour =} \StringTok{'red'}\NormalTok{, }\DataTypeTok{linetype=}\StringTok{"dotted"}\NormalTok{) }\OperatorTok{+}
\StringTok{        }\KeywordTok{geom_quantile}\NormalTok{(}\DataTypeTok{quantiles =} \FloatTok{0.5}\NormalTok{, }\DataTypeTok{colour =} \StringTok{'green'}\NormalTok{) }
\end{Highlighting}
\end{Shaded}

\begin{verbatim}
## Smoothing formula not specified. Using: y ~ x
## Smoothing formula not specified. Using: y ~ x
\end{verbatim}

\includegraphics{Material_files/figure-latex/unnamed-chunk-3-1.pdf}

We can interpret the causal relationship in quantile regression only
under rank invariant condition. This requires that individuals have
always the same ranking in the distribution of \(Y(X)\) no matter the
\(X\). This is difficult to verify or believe in actual applications,
but feasible in theory.

Under the rank invariant condition \(\beta_{\tau}\) can be interpreted
as the effect on \(Y\) of an increase of one unit of \(X\) among
entities at rank \(\tau\) in the distribution \(Y|X=x\).

\section{Replication of a paper using quantile
regression}\label{replication-of-a-paper-using-quantile-regression}

We are going to replicate the quantile regression procedure by
\citep{abrevaya2002effects}. In the
\href{https://link.springer.com/article/10.1007/s001810000052}{paper}
the author investigates the impact of different demographic
characteristics and maternal behaviour on the weight at birth in the
United States in 1992 and 1996. Why is this relevant:

\begin{itemize}
\tightlist
\item
  There is a correlation between health problems after birth for
  underweight chidren
\item
  There might be a relation with labor market participation and
  educational attaintment later in life
\item
  There are incentives to create specific programs to deal with
  underweight children; it is important to understand such behaviours
\end{itemize}

\subsubsection{Data:}\label{data}

In order to get the data we access the following
\href{https://www.nber.org/data/vital-statistics-natality-data.html}{link}.
Here you can download the `NCHS' Vital Statistics Natality Birth Data',
which is the data used in the paper. We are using only the 1992 and 1996
waves. To download the data follow this link
\href{https://www.nber.org/natality/1992/natl1992.csv.zip}{for the zip
CSV file of 1992}. One important thing to do when analizing the data is
understand your data before the actual analysis. Before you start you
take a minute or two to consider:

\begin{itemize}
\tightlist
\item
  What is the data?
\item
  Where does it come from? What is the universe, the population and what
  is your sample.
\item
  What is the shape, format of your data? Do I have access to a data
  dictionary?
\end{itemize}

In this case many of this questions are available in the documentation
that is provided {[}at the following link{]}
(\url{https://www.nber.org/natality/1992/natl1992.pdf}) detailing every
variable in the dataset. From the documentation we can see the kind of
information (a glimpse of the amount of variables) and the data counts
(\(4'069'428\) observations). An indicator of the size of the data is
the size of the file: the zip file is \emph{156 Mb} and the uncompressed
version of the CSV \emph{2.07 GB}! Even if this does not seem much,
consider that all this data is stored in the cache of your RAM memory,
and it can easily slow down even recent machines. For this reason it is
better to read in only the columns that we are interested in. This
requires reading the manual and a prior inspection of a subset of the
data, which allows you to know the structure, column types and other
properties. The most efficient function to open plain text files is
\emph{`fread'} from the \emph{`data.table'} package
\citep{R-data.table}. First let's investigate the data. The first thing
to do is to load the required libraries int the current session:

\begin{itemize}
\tightlist
\item
  \emph{`data.table'} to open the data
\item
  \emph{`quantreg'} to perform the quantile regression estimation
\item
  \emph{`stargazer'} to export the results in latex
\item
  \emph{`dplyr'} to manipulate the data to create the summary statistics
\end{itemize}

\begin{Shaded}
\begin{Highlighting}[]
\KeywordTok{library}\NormalTok{(data.table)}
\KeywordTok{library}\NormalTok{(quantreg)}
\KeywordTok{library}\NormalTok{(stargazer)}
\KeywordTok{library}\NormalTok{(dplyr)}
\KeywordTok{library}\NormalTok{(kableExtra)}
\end{Highlighting}
\end{Shaded}

\begin{Shaded}
\begin{Highlighting}[]
\NormalTok{path_source <-}\StringTok{ "YOUR PATH GOES HERE"}
\CommentTok{# for macOS and linux: use / in your path to data}
\CommentTok{# for Win: rembember to use \textbackslash{}\textbackslash{} instead of /}
\end{Highlighting}
\end{Shaded}

\begin{Shaded}
\begin{Highlighting}[]
\NormalTok{data <-}\StringTok{ }\KeywordTok{fread}\NormalTok{(path_source, }\DataTypeTok{nrows =} \KeywordTok{c}\NormalTok{(}\DecValTok{100}\NormalTok{))}
\KeywordTok{head}\NormalTok{(data[,}\KeywordTok{c}\NormalTok{(}\DecValTok{35}\OperatorTok{:}\DecValTok{41}\NormalTok{)])}
\end{Highlighting}
\end{Shaded}

\begin{verbatim}
##    mage12 mage8 ormoth orracem mraceimp mrace mrace3
## 1:      9     4      0       7       NA     2      3
## 2:      9     4      0       7       NA     2      3
## 3:      8     3      0       7       NA     2      3
## 4:      8     3      0       6       NA     1      1
## 5:      5     2      0       6       NA     1      1
## 6:      9     4      0       6       NA     1      1
\end{verbatim}

\begin{Shaded}
\begin{Highlighting}[]
\NormalTok{type_of_data <-}\StringTok{ }\NormalTok{data }\OperatorTok\StringTok{ }\KeywordTok{summarise_all}\NormalTok{(typeof)}
\NormalTok{type_of_data[}\DecValTok{35}\OperatorTok{:}\DecValTok{41}\NormalTok{]}
\end{Highlighting}
\end{Shaded}

\begin{verbatim}
##    mage12   mage8  ormoth orracem mraceimp   mrace  mrace3
## 1 integer integer integer integer  logical integer integer
\end{verbatim}

Now that we have had a look at the content of the database, let's import
the data and clean it. To import only the relevant variables of the
paper we create a list containing all the relevant variables for the
estimation. I used the codebook to construct this list. Then we use this
list within `fread()' to import solely the desired columns.

\begin{Shaded}
\begin{Highlighting}[]
\NormalTok{desired <-}\StringTok{ }\KeywordTok{c}\NormalTok{(}\StringTok{"birmon"}\NormalTok{,}\StringTok{"mrace3"}\NormalTok{,}\StringTok{"dmage"}\NormalTok{,}\StringTok{"dbirwt"}\NormalTok{,}\StringTok{"dplural"}\NormalTok{,}\StringTok{"stnatexp"}\NormalTok{,}
             \StringTok{"mraceimp"}\NormalTok{,}\StringTok{"dmarimp"}\NormalTok{,}\StringTok{"mageimp"}\NormalTok{,}\StringTok{"cseximp"}\NormalTok{,}\StringTok{"dmar"}\NormalTok{,}\StringTok{"meduc6"}\NormalTok{, }
             \StringTok{"wtgain"}\NormalTok{, }\StringTok{"mpre5"}\NormalTok{, }\StringTok{"tobacco"}\NormalTok{,}\StringTok{"cigar"}\NormalTok{, }\StringTok{"csex"}\NormalTok{, }\StringTok{"plurimp"}\NormalTok{, }\StringTok{"restatus"}\NormalTok{)}

\NormalTok{data <-}\StringTok{ }\KeywordTok{fread}\NormalTok{(path_source, }\DataTypeTok{select =}\NormalTok{ desired)}
\end{Highlighting}
\end{Shaded}

Following the indications of the paper:

\begin{quote}
To cut down he sample size, we have decided to use only births occurring
in June {[}\ldots{}{]} There is no evidence that suggest that the June
samplediffers in any meaningful way to the full sample. The sample was
further limited to singleton births and mothers who were either white or
black, between ages 18 and 45, and residents of the United States.
Observations for which there was missing information on any relevant
variable were also dropped. Unfortunately, all births occurring in
California, Indiana, New York, and South Dakota had to be dropped from
the sample since these states either did not asked a question about
smoking during pregnancy or did not ask it in a form compatible with
NHCS standards\ldots{}
\end{quote}

We need to apply the following filters:

\begin{itemize}
\tightlist
\item
  Remove all obs. in months different than the sixth (June)
\item
  Remove all non white or non black mothers
\item
  Remove all obs. which age is not in {[}18,45{]} group
\item
  Remove the non stated weights
\item
  Remove all the obs. from California, New York, Indiana and South
  Dakota
\end{itemize}

\begin{Shaded}
\begin{Highlighting}[]
\NormalTok{data <-}\StringTok{ }\NormalTok{data[}\KeywordTok{which}\NormalTok{(data}\OperatorTok{$}\NormalTok{restatus}\OperatorTok{!=}\DecValTok{4}\NormalTok{),]}
\NormalTok{data <-}\StringTok{ }\NormalTok{data[}\KeywordTok{which}\NormalTok{(data}\OperatorTok{$}\NormalTok{birmon}\OperatorTok{==}\DecValTok{6}\NormalTok{),]}
\NormalTok{data <-}\StringTok{ }\NormalTok{data[}\KeywordTok{which}\NormalTok{(data}\OperatorTok{$}\NormalTok{mrace3}\OperatorTok{!=}\DecValTok{2}\NormalTok{),]}
\NormalTok{data <-}\StringTok{ }\NormalTok{data[}\KeywordTok{which}\NormalTok{(data}\OperatorTok{$}\NormalTok{dmage}\OperatorTok{>}\DecValTok{17}\NormalTok{),]}
\NormalTok{data <-}\StringTok{ }\NormalTok{data[}\KeywordTok{which}\NormalTok{(data}\OperatorTok{$}\NormalTok{dmage}\OperatorTok{<}\DecValTok{46}\NormalTok{),]}
\NormalTok{data <-}\StringTok{ }\NormalTok{data[}\KeywordTok{which}\NormalTok{(data}\OperatorTok{$}\NormalTok{dbirwt}\OperatorTok{!=}\DecValTok{9999}\NormalTok{),]}
\NormalTok{data <-}\StringTok{ }\NormalTok{data[}\KeywordTok{which}\NormalTok{(data}\OperatorTok{$}\NormalTok{dplural}\OperatorTok{==}\DecValTok{1}\NormalTok{),]}
\NormalTok{data <-}\StringTok{ }\NormalTok{data[}\KeywordTok{which}\NormalTok{(}\OperatorTok{!}\NormalTok{(data}\OperatorTok{$}\NormalTok{stnatexp }\OperatorTok\StringTok{ }\KeywordTok{c}\NormalTok{(}\StringTok{"05"}\NormalTok{,}\StringTok{"33"}\NormalTok{,}\StringTok{"34"}\NormalTok{,}\StringTok{"15"}\NormalTok{,}\StringTok{"43"}\NormalTok{))),]}
\end{Highlighting}
\end{Shaded}

Moreover, we remove the missing observations. One particularity of many
of the variables of the database is that the missing values are coded,
meaning that are not representes by
\texttt{\textquotesingle{}NA\textquotesingle{}} but by a code that
changes in each variable. We are going to remove also the missing from
those variables:

\begin{Shaded}
\begin{Highlighting}[]
\NormalTok{data <-}\StringTok{ }\NormalTok{data[}\KeywordTok{which}\NormalTok{(}\OperatorTok{!}\NormalTok{(data}\OperatorTok{$}\NormalTok{plurimp }\OperatorTok\StringTok{ }\KeywordTok{c}\NormalTok{(}\DecValTok{1}\NormalTok{))),]}
\NormalTok{data <-}\StringTok{ }\NormalTok{data[}\KeywordTok{which}\NormalTok{(}\OperatorTok{!}\NormalTok{(data}\OperatorTok{$}\NormalTok{mraceimp }\OperatorTok\StringTok{ }\KeywordTok{c}\NormalTok{(}\DecValTok{1}\NormalTok{))),]}
\NormalTok{data <-}\StringTok{ }\NormalTok{data[}\KeywordTok{which}\NormalTok{(}\OperatorTok{!}\NormalTok{(data}\OperatorTok{$}\NormalTok{dmarimp }\OperatorTok\StringTok{ }\KeywordTok{c}\NormalTok{(}\DecValTok{1}\NormalTok{))),]}
\NormalTok{data <-}\StringTok{ }\NormalTok{data[}\KeywordTok{which}\NormalTok{(}\OperatorTok{!}\NormalTok{(data}\OperatorTok{$}\NormalTok{mageimp }\OperatorTok\StringTok{ }\KeywordTok{c}\NormalTok{(}\DecValTok{1}\NormalTok{))),]}
\NormalTok{data <-}\StringTok{ }\NormalTok{data[}\KeywordTok{which}\NormalTok{(}\OperatorTok{!}\NormalTok{(data}\OperatorTok{$}\NormalTok{cseximp }\OperatorTok\StringTok{ }\KeywordTok{c}\NormalTok{(}\DecValTok{1}\NormalTok{))),]}
\NormalTok{toKeep <-}\StringTok{ }\KeywordTok{c}\NormalTok{(}\StringTok{"dbirwt"}\NormalTok{, }\StringTok{"mrace3"}\NormalTok{,}\StringTok{"dmar"}\NormalTok{,}\StringTok{"dmage"}\NormalTok{,}\StringTok{"meduc6"}\NormalTok{, }\StringTok{"wtgain"}\NormalTok{, }\StringTok{"mpre5"}\NormalTok{, }\StringTok{"tobacco"}\NormalTok{,}\StringTok{"cigar"}\NormalTok{, }\StringTok{"csex"}\NormalTok{)}
\NormalTok{data <-}\StringTok{ }\KeywordTok{as.data.frame}\NormalTok{(data)}
\NormalTok{data <-}\StringTok{ }\NormalTok{data[,toKeep]}
\NormalTok{data <-}\StringTok{ }\NormalTok{data[}\KeywordTok{which}\NormalTok{(data}\OperatorTok{$}\NormalTok{dbirwt}\OperatorTok{!=}\DecValTok{9999}\NormalTok{),]}
\NormalTok{data <-}\StringTok{ }\NormalTok{data[}\KeywordTok{which}\NormalTok{(data}\OperatorTok{$}\NormalTok{wtgain}\OperatorTok{!=}\DecValTok{99}\NormalTok{),]}
\NormalTok{data <-}\StringTok{ }\NormalTok{data[}\KeywordTok{which}\NormalTok{(data}\OperatorTok{$}\NormalTok{dmage}\OperatorTok{!=}\DecValTok{99}\NormalTok{),]}
\NormalTok{data <-}\StringTok{ }\NormalTok{data[}\KeywordTok{which}\NormalTok{(data}\OperatorTok{$}\NormalTok{meduc6}\OperatorTok{!=}\DecValTok{6}\NormalTok{),]}
\NormalTok{data <-}\StringTok{ }\NormalTok{data[}\KeywordTok{which}\NormalTok{(data}\OperatorTok{$}\NormalTok{mpre5}\OperatorTok{!=}\DecValTok{5}\NormalTok{),]}
\NormalTok{data <-}\StringTok{ }\NormalTok{data[}\KeywordTok{which}\NormalTok{(data}\OperatorTok{$}\NormalTok{tobacco}\OperatorTok{!=}\DecValTok{9}\NormalTok{),]}
\NormalTok{data <-}\StringTok{ }\NormalTok{data[}\KeywordTok{which}\NormalTok{(data}\OperatorTok{$}\NormalTok{cigar}\OperatorTok{!=}\DecValTok{99}\NormalTok{),]}
\NormalTok{data <-}\StringTok{ }\NormalTok{data[}\KeywordTok{which}\NormalTok{(data}\OperatorTok{$}\NormalTok{wtgain }\OperatorTok{!=}\DecValTok{99}\NormalTok{),]}
\end{Highlighting}
\end{Shaded}

For the categorical variables (i.e.~education of the mother), we create
dummy variables. We keep the contrast (levels of reference) according to
the paper. In this way we will be able to compare the results.

\begin{Shaded}
\begin{Highlighting}[]
\NormalTok{data}\OperatorTok{$}\NormalTok{black <-}\StringTok{ }\KeywordTok{ifelse}\NormalTok{(data}\OperatorTok{$}\NormalTok{mrace3}\OperatorTok{==}\DecValTok{3}\NormalTok{,}\DecValTok{1}\NormalTok{,}\DecValTok{0}\NormalTok{)}
\NormalTok{data}\OperatorTok{$}\NormalTok{married <-}\StringTok{ }\KeywordTok{ifelse}\NormalTok{(data}\OperatorTok{$}\NormalTok{dmar}\OperatorTok{==}\DecValTok{1}\NormalTok{,}\DecValTok{1}\NormalTok{,}\DecValTok{0}\NormalTok{)}
\NormalTok{data}\OperatorTok{$}\NormalTok{agesq <-}\StringTok{ }\NormalTok{(data}\OperatorTok{$}\NormalTok{dmage)}\OperatorTok{^}\DecValTok{2}
\NormalTok{data}\OperatorTok{$}\NormalTok{hsgrad <-}\StringTok{ }\KeywordTok{ifelse}\NormalTok{(data}\OperatorTok{$}\NormalTok{meduc6}\OperatorTok{==}\DecValTok{3}\NormalTok{,}\DecValTok{1}\NormalTok{,}\DecValTok{0}\NormalTok{)}
\NormalTok{data}\OperatorTok{$}\NormalTok{somecoll <-}\StringTok{ }\KeywordTok{ifelse}\NormalTok{(data}\OperatorTok{$}\NormalTok{meduc6}\OperatorTok{==}\DecValTok{4}\NormalTok{,}\DecValTok{1}\NormalTok{,}\DecValTok{0}\NormalTok{)}
\NormalTok{data}\OperatorTok{$}\NormalTok{collgrad <-}\StringTok{ }\KeywordTok{ifelse}\NormalTok{(data}\OperatorTok{$}\NormalTok{meduc6}\OperatorTok{==}\DecValTok{5}\NormalTok{,}\DecValTok{1}\NormalTok{,}\DecValTok{0}\NormalTok{)}
\NormalTok{data}\OperatorTok{$}\NormalTok{natal2 <-}\StringTok{ }\KeywordTok{ifelse}\NormalTok{(data}\OperatorTok{$}\NormalTok{mpre5}\OperatorTok{==}\DecValTok{2}\NormalTok{,}\DecValTok{1}\NormalTok{,}\DecValTok{0}\NormalTok{)}
\NormalTok{data}\OperatorTok{$}\NormalTok{natal3 <-}\StringTok{ }\KeywordTok{ifelse}\NormalTok{(data}\OperatorTok{$}\NormalTok{mpre5}\OperatorTok{==}\DecValTok{3}\NormalTok{,}\DecValTok{1}\NormalTok{,}\DecValTok{0}\NormalTok{)}
\NormalTok{data}\OperatorTok{$}\NormalTok{novisit <-}\StringTok{ }\KeywordTok{ifelse}\NormalTok{(data}\OperatorTok{$}\NormalTok{mpre5}\OperatorTok{==}\DecValTok{4}\NormalTok{,}\DecValTok{1}\NormalTok{,}\DecValTok{0}\NormalTok{)}
\NormalTok{data}\OperatorTok{$}\NormalTok{nosmoke <-}\StringTok{ }\KeywordTok{ifelse}\NormalTok{(data}\OperatorTok{$}\NormalTok{tobacco}\OperatorTok{==}\DecValTok{2}\NormalTok{,}\DecValTok{1}\NormalTok{,}\DecValTok{0}\NormalTok{)}
\NormalTok{data}\OperatorTok{$}\NormalTok{boy <-}\StringTok{ }\KeywordTok{ifelse}\NormalTok{(data}\OperatorTok{$}\NormalTok{csex}\OperatorTok{==}\DecValTok{1}\NormalTok{,}\DecValTok{1}\NormalTok{,}\DecValTok{0}\NormalTok{)}
\end{Highlighting}
\end{Shaded}

We also relabel variables to match those in the paper:

\begin{Shaded}
\begin{Highlighting}[]
\NormalTok{finalVars <-}\StringTok{ }\KeywordTok{c}\NormalTok{(}\StringTok{"dbirwt"}\NormalTok{,}\StringTok{"black"}\NormalTok{, }\StringTok{"married"}\NormalTok{, }\StringTok{"dmage"}\NormalTok{, }\StringTok{"agesq"}\NormalTok{, }\StringTok{"hsgrad"}\NormalTok{, }\StringTok{"somecoll"}\NormalTok{, }\StringTok{"collgrad"}\NormalTok{, }\StringTok{"wtgain"}\NormalTok{, }\StringTok{"natal2"}\NormalTok{, }\StringTok{"natal3"}\NormalTok{, }\StringTok{"novisit"}\NormalTok{, }\StringTok{"nosmoke"}\NormalTok{, }\StringTok{"cigar"}\NormalTok{, }\StringTok{"boy"}\NormalTok{)}
\NormalTok{data <-}\StringTok{ }\NormalTok{data[, finalVars]}
\KeywordTok{names}\NormalTok{(data) <-}\StringTok{ }\KeywordTok{c}\NormalTok{(}\StringTok{"birwt"}\NormalTok{,}\StringTok{"black"}\NormalTok{, }\StringTok{"married"}\NormalTok{, }\StringTok{"age"}\NormalTok{, }\StringTok{"agesq"}\NormalTok{, }\StringTok{"hsgrad"}\NormalTok{, }\StringTok{"somecoll"}\NormalTok{, }\StringTok{"collgrad"}\NormalTok{,}\StringTok{"wtgain"}\NormalTok{, }\StringTok{"natal2"}\NormalTok{, }\StringTok{"natal3"}\NormalTok{, }\StringTok{"novisit"}\NormalTok{, }\StringTok{"nosmoke"}\NormalTok{, }\StringTok{"cigar"}\NormalTok{, }\StringTok{"boy"}\NormalTok{) }
\end{Highlighting}
\end{Shaded}

Now let's make a summary table. We are going to use the \texttt{dplyr}
package \citep{R-dplyr} for the data transformation and the
\texttt{stargazer} package \citep{R-stargazer} to nicely export the
results. We construct the columns for all selected variables. To store
the results it is convenient to save tables in ( \LaTeX ) format, so you
can use them directly form that folder into your paper, or copy the
result from them.

\begin{Shaded}
\begin{Highlighting}[]
\NormalTok{desc_stats <-}\StringTok{ }\KeywordTok{round}\NormalTok{(}\KeywordTok{as.data.frame}\NormalTok{(}\KeywordTok{cbind}\NormalTok{(}\KeywordTok{t}\NormalTok{(data }\OperatorTok\StringTok{ }\KeywordTok{summarise_all}\NormalTok{(mean)),}\KeywordTok{t}\NormalTok{(data }\OperatorTok\StringTok{ }\KeywordTok{summarise_all}\NormalTok{(sd)))),}\DecValTok{3}\NormalTok{)}
\KeywordTok{names}\NormalTok{(desc_stats) <-}\StringTok{ }\KeywordTok{c}\NormalTok{(}\StringTok{"Mean"}\NormalTok{, }\StringTok{"Standard Deviation"}\NormalTok{)}
\end{Highlighting}
\end{Shaded}

\begin{Shaded}
\begin{Highlighting}[]
\NormalTok{stargazer}\OperatorTok{::}\KeywordTok{stargazer}\NormalTok{(desc_stats,}
\DataTypeTok{type=}\KeywordTok{ifelse}\NormalTok{(knitr}\OperatorTok{::}\KeywordTok{is_latex_output}\NormalTok{(),}\StringTok{"latex"}\NormalTok{,}\StringTok{"html"}\NormalTok{),}
\DataTypeTok{label=}\NormalTok{knitr}\OperatorTok{::}\NormalTok{opts_current}\OperatorTok{$}\KeywordTok{get}\NormalTok{(}\StringTok{"label"}\NormalTok{),}
\DataTypeTok{title=}\StringTok{"Summary Statistics 1992"}\NormalTok{, }\DataTypeTok{summary =} \OtherTok{FALSE}\NormalTok{, }\DataTypeTok{out=}\StringTok{"./images/summary_table.tex"}\NormalTok{)}
\end{Highlighting}
\end{Shaded}

\% Table created by stargazer v.5.2.2 by Marek Hlavac, Harvard
University. E-mail: hlavac at fas.harvard.edu \% Date and time: Mon, Oct
07, 2019 - 19:26:23

\begin{table}[!htbp] \centering 
  \caption{Summary Statistics 1992} 
  \label{tab:unnamed-chunk-14} 
\begin{tabular}{@{\extracolsep{5pt}} ccc} 
\\[-1.8ex]\hline 
\hline \\[-1.8ex] 
 & Mean & Standard Deviation \\ 
\hline \\[-1.8ex] 
birwt & $3,388.683$ & $571.511$ \\ 
black & $0.163$ & $0.370$ \\ 
married & $0.752$ & $0.432$ \\ 
age & $26.960$ & $5.434$ \\ 
agesq & $756.389$ & $303.877$ \\ 
hsgrad & $0.394$ & $0.489$ \\ 
somecoll & $0.232$ & $0.422$ \\ 
collgrad & $0.211$ & $0.408$ \\ 
wtgain & $30.760$ & $12.245$ \\ 
natal2 & $0.158$ & $0.365$ \\ 
natal3 & $0.026$ & $0.161$ \\ 
novisit & $0.009$ & $0.096$ \\ 
nosmoke & $0.830$ & $0.376$ \\ 
cigar & $2.139$ & $5.737$ \\ 
boy & $0.513$ & $0.500$ \\ 
\hline \\[-1.8ex] 
\end{tabular} 
\end{table}

Now let's calculate the quantile regression:

How does the command rq() work? An easy way to access the help file of a
command, just put a question mark before it in the console, i.e.~if you
want to collect information on the \emph{mean()} function you would type
\texttt{?mean}. For our quantile regression, we are going to use the
function \texttt{rq()} from the `quantreg' package.

From the help file we can see that the principal inputs of the function
are `formula' (the relationship to evaluate), the `tau' (the vector of
quantiles), and the `data', which is a dataframe containing the
information. Regarding the data it requires a specific type of object, a
\emph{`data.frame'}, and also specifies that if we have factors among
our variables, it is important to provide a vector with the contrast
levels. We do not have to provide it since we already constructed our
dummies for such purpose. We are only missing the formula and the
quantile vector.

For the moment we are going to construct the formula:

\begin{Shaded}
\begin{Highlighting}[]
\NormalTok{Y <-}\StringTok{ "birwt"}
\NormalTok{X <-}\StringTok{ }\KeywordTok{paste}\NormalTok{(}\KeywordTok{names}\NormalTok{(data[,}\OperatorTok{-}\DecValTok{1}\NormalTok{]),}\DataTypeTok{collapse =} \StringTok{" + "}\NormalTok{)}
\NormalTok{formula_qr <-}\StringTok{ }\KeywordTok{as.formula}\NormalTok{(}\KeywordTok{paste}\NormalTok{(Y, }\StringTok{" ~ "}\NormalTok{, X))}
\NormalTok{formula_qr}
\end{Highlighting}
\end{Shaded}

\begin{verbatim}
## birwt ~ black + married + age + agesq + hsgrad + somecoll + collgrad + 
##     wtgain + natal2 + natal3 + novisit + nosmoke + cigar + boy
\end{verbatim}

And then we construct the vector of quantiles:

\begin{Shaded}
\begin{Highlighting}[]
\NormalTok{quantiles_table <-}\StringTok{ }\KeywordTok{c}\NormalTok{(}\FloatTok{0.1}\NormalTok{,}\FloatTok{0.25}\NormalTok{,}\FloatTok{0.5}\NormalTok{,}\FloatTok{0.75}\NormalTok{,}\FloatTok{0.9}\NormalTok{)}
\end{Highlighting}
\end{Shaded}

As in the main slides, if the number of observations is not large
enough, we can use the simplex method, but given that we have more than
a hundred thousand observations we are going to use the
\emph{`Frish-Newton' interior point} method. We specify this option in
the command options including the \texttt{method\ =\ "fn"} statement.
After the calculation we will have an object of class `rq' or `rqs',
depending on the number of quantiles specified; this might be relevant
since some commands might behave differently if operated over this
object. For example, the command \texttt{summary}, that is often used to
get the summary statistics of a `data.frame', when applied to a `rq' or
`rqs' object returns the summary of the fit of the QR.

\begin{Shaded}
\begin{Highlighting}[]
\NormalTok{quant_reg_res1992 <-}\StringTok{ }\KeywordTok{rq}\NormalTok{(formula_qr,}\DataTypeTok{tau =}\NormalTok{ quantiles_table, }\DataTypeTok{data =}\NormalTok{ data, }\DataTypeTok{method =} \StringTok{'fn'}\NormalTok{)}
\NormalTok{qr_results <-}\StringTok{ }\KeywordTok{summary.rqs}\NormalTok{(quant_reg_res1992)}
\NormalTok{qr_results}
\end{Highlighting}
\end{Shaded}

\begin{verbatim}
## 
## Call: rq(formula = formula_qr, tau = quantiles_table, data = data, 
##     method = "fn")
## 
## tau: [1] 0.1
## 
## Coefficients:
##             Value      Std. Error t value    Pr(>|t|)  
## (Intercept) 1556.90509   58.45879   26.63252    0.00000
## black       -253.65558    7.91665  -32.04078    0.00000
## married       73.32313    6.55406   11.18744    0.00000
## age           45.17747    4.27348   10.57159    0.00000
## agesq         -0.78109    0.07630  -10.23749    0.00000
## hsgrad        28.57972    7.31354    3.90778    0.00009
## somecoll      49.49438    8.22498    6.01757    0.00000
## collgrad      82.64542    8.79916    9.39242    0.00000
## wtgain        11.77551    0.16692   70.54735    0.00000
## natal2         6.17644    6.55496    0.94225    0.34606
## natal3        39.43374   15.11504    2.60891    0.00908
## novisit     -388.58389   44.08937   -8.81355    0.00000
## nosmoke      170.75013   11.15951   15.30087    0.00000
## cigar         -3.80301    0.63954   -5.94644    0.00000
## boy           87.76342    4.50293   19.49028    0.00000
## 
## Call: rq(formula = formula_qr, tau = quantiles_table, data = data, 
##     method = "fn")
## 
## tau: [1] 0.25
## 
## Coefficients:
##             Value      Std. Error t value    Pr(>|t|)  
## (Intercept) 2020.48850   37.98519   53.19147    0.00000
## black       -216.79384    4.64082  -46.71456    0.00000
## married       55.18092    4.23478   13.03042    0.00000
## age           36.72168    2.75656   13.32156    0.00000
## agesq         -0.59142    0.04893  -12.08786    0.00000
## hsgrad        25.07794    4.76923    5.25828    0.00000
## somecoll      40.54889    5.23706    7.74268    0.00000
## collgrad      57.60057    5.76602    9.98967    0.00000
## wtgain         9.90517    0.11880   83.37620    0.00000
## natal2         2.83300    4.17057    0.67928    0.49696
## natal3        12.67925    9.60040    1.32070    0.18660
## novisit     -196.45154   24.28562   -8.08921    0.00000
## nosmoke      171.83979    7.21094   23.83042    0.00000
## cigar         -3.51950    0.46625   -7.54859    0.00000
## boy          109.56824    2.93365   37.34874    0.00000
## 
## Call: rq(formula = formula_qr, tau = quantiles_table, data = data, 
##     method = "fn")
## 
## tau: [1] 0.5
## 
## Coefficients:
##             Value      Std. Error t value    Pr(>|t|)  
## (Intercept) 2377.88346   31.58611   75.28256    0.00000
## black       -199.07093    3.82874  -51.99386    0.00000
## married       50.56335    3.57758   14.13338    0.00000
## age           34.63588    2.29095   15.11860    0.00000
## agesq         -0.52963    0.04064  -13.03315    0.00000
## hsgrad        17.54254    3.82369    4.58786    0.00000
## somecoll      31.69835    4.44505    7.13116    0.00000
## collgrad      36.83945    4.81609    7.64925    0.00000
## wtgain         9.13414    0.10254   89.08206    0.00000
## natal2        -4.47675    3.67308   -1.21880    0.22292
## natal3         3.96735    7.23390    0.54844    0.58339
## novisit     -147.23008   15.30005   -9.62285    0.00000
## nosmoke      158.82091    6.10852   25.99989    0.00000
## cigar         -3.90210    0.39009  -10.00299    0.00000
## boy          129.12756    2.55257   50.58719    0.00000
## 
## Call: rq(formula = formula_qr, tau = quantiles_table, data = data, 
##     method = "fn")
## 
## tau: [1] 0.75
## 
## Coefficients:
##             Value      Std. Error t value    Pr(>|t|)  
## (Intercept) 2715.22798   34.34314   79.06174    0.00000
## black       -192.40481    4.26347  -45.12869    0.00000
## married       42.45607    4.08114   10.40300    0.00000
## age           31.90923    2.50037   12.76181    0.00000
## agesq         -0.43958    0.04427   -9.92928    0.00000
## hsgrad        15.02372    4.38866    3.42330    0.00062
## somecoll      26.97497    4.98902    5.40687    0.00000
## collgrad      16.26961    5.42208    3.00062    0.00269
## wtgain         8.83829    0.11772   75.08147    0.00000
## natal2        -0.54374    4.07482   -0.13344    0.89385
## natal3        -6.23893    8.86964   -0.70340    0.48181
## novisit     -126.64150   16.13168   -7.85049    0.00000
## nosmoke      153.22724    6.65373   23.02876    0.00000
## cigar         -4.46120    0.40831  -10.92596    0.00000
## boy          142.40192    2.86197   49.75669    0.00000
## 
## Call: rq(formula = formula_qr, tau = quantiles_table, data = data, 
##     method = "fn")
## 
## tau: [1] 0.9
## 
## Coefficients:
##             Value      Std. Error t value    Pr(>|t|)  
## (Intercept) 2967.68273   47.25148   62.80613    0.00000
## black       -182.08652    5.68366  -32.03682    0.00000
## married       38.55994    5.42803    7.10386    0.00000
## age           33.78764    3.45080    9.79126    0.00000
## agesq         -0.43555    0.06158   -7.07260    0.00000
## hsgrad        13.35546    5.86831    2.27586    0.02286
## somecoll      18.63983    6.67219    2.79366    0.00521
## collgrad      -4.66343    7.47727   -0.62368    0.53284
## wtgain         8.57592    0.15653   54.78798    0.00000
## natal2         3.79637    5.63320    0.67393    0.50036
## natal3       -25.49765   10.73838   -2.37444    0.01758
## novisit     -101.65891   17.28236   -5.88223    0.00000
## nosmoke      150.21941    9.44534   15.90408    0.00000
## cigar         -5.16788    0.60668   -8.51835    0.00000
## boy          153.58224    3.83777   40.01857    0.00000
\end{verbatim}

One important consideration is the kind of errors that the procedure is
calculating when running the summary.

Nevertheless this kind of results are not easy to handle and manage, and
is desireable to have a summary table like the one of the paper or to
summarize the results in just one table. We have seen already that is
possible to save the results in LaTeX, now we are going to save them in
TXT format, which might be usefull in many cases. To this end, we select
the first and second column of the results. If in this case the list
contains only five items, is still acceptable to do it line by line. If
an operation has more elements and is used often it is better to write a
function to save time and avoid copypaste mistakes.

\begin{Shaded}
\begin{Highlighting}[]
\NormalTok{tab_res <-}\StringTok{ }\KeywordTok{as.data.frame}\NormalTok{(}\KeywordTok{cbind}\NormalTok{(qr_results[[}\DecValTok{1}\NormalTok{]]}\OperatorTok{$}\NormalTok{coefficients[,}\DecValTok{1}\NormalTok{], }
\NormalTok{                 qr_results[[}\DecValTok{2}\NormalTok{]]}\OperatorTok{$}\NormalTok{coefficients[,}\DecValTok{1}\NormalTok{], }
\NormalTok{                 qr_results[[}\DecValTok{3}\NormalTok{]]}\OperatorTok{$}\NormalTok{coefficients[,}\DecValTok{1}\NormalTok{], }
\NormalTok{                 qr_results[[}\DecValTok{4}\NormalTok{]]}\OperatorTok{$}\NormalTok{coefficients[,}\DecValTok{1}\NormalTok{], }
\NormalTok{                 qr_results[[}\DecValTok{5}\NormalTok{]]}\OperatorTok{$}\NormalTok{coefficients[,}\DecValTok{1}\NormalTok{]))}
\KeywordTok{names}\NormalTok{(tab_res) <-}\StringTok{ }\KeywordTok{paste0}\NormalTok{(}\StringTok{"Tau_"}\NormalTok{,quantiles_table,}\StringTok{"_beta"}\NormalTok{)}

\NormalTok{tab_ES <-}\StringTok{ }\KeywordTok{as.data.frame}\NormalTok{(}\KeywordTok{cbind}\NormalTok{(qr_results[[}\DecValTok{1}\NormalTok{]]}\OperatorTok{$}\NormalTok{coefficients[,}\DecValTok{2}\NormalTok{], }
\NormalTok{                 qr_results[[}\DecValTok{2}\NormalTok{]]}\OperatorTok{$}\NormalTok{coefficients[,}\DecValTok{2}\NormalTok{], }
\NormalTok{                 qr_results[[}\DecValTok{3}\NormalTok{]]}\OperatorTok{$}\NormalTok{coefficients[,}\DecValTok{2}\NormalTok{], }
\NormalTok{                 qr_results[[}\DecValTok{4}\NormalTok{]]}\OperatorTok{$}\NormalTok{coefficients[,}\DecValTok{2}\NormalTok{], }
\NormalTok{                 qr_results[[}\DecValTok{5}\NormalTok{]]}\OperatorTok{$}\NormalTok{coefficients[,}\DecValTok{2}\NormalTok{]))}

\KeywordTok{names}\NormalTok{(tab_ES) <-}\StringTok{ }\KeywordTok{paste0}\NormalTok{(}\StringTok{"Tau_"}\NormalTok{,quantiles_table,}\StringTok{"_SE"}\NormalTok{)}

\NormalTok{results <-}\StringTok{ }\KeywordTok{as.data.frame}\NormalTok{(}\KeywordTok{cbind}\NormalTok{(tab_res,tab_ES))}
\NormalTok{results <-}\StringTok{ }\NormalTok{results[,}\KeywordTok{sort}\NormalTok{(}\KeywordTok{names}\NormalTok{(results))]}
\NormalTok{results <-}\StringTok{ }\NormalTok{results[}\OperatorTok{-}\DecValTok{1}\NormalTok{,]}
\end{Highlighting}
\end{Shaded}

\begin{Shaded}
\begin{Highlighting}[]
\NormalTok{stargazer}\OperatorTok{::}\KeywordTok{stargazer}\NormalTok{(results, }\DataTypeTok{type=}\StringTok{'text'}\NormalTok{, }\DataTypeTok{out=}\StringTok{"./images/qr_res1992.tex"}\NormalTok{, }\DataTypeTok{summary =} \OtherTok{FALSE}\NormalTok{)}
\end{Highlighting}
\end{Shaded}

\begin{verbatim}
## 
## ====================================================================================================================================
##          Tau_0.1_beta Tau_0.1_SE Tau_0.25_beta Tau_0.25_SE Tau_0.5_beta Tau_0.5_SE Tau_0.75_beta Tau_0.75_SE Tau_0.9_beta Tau_0.9_SE
## ------------------------------------------------------------------------------------------------------------------------------------
## black      -253.656     7.917      -216.794       4.641      -199.071     3.829      -192.405       4.263      -182.087     5.684   
## married     73.323      6.554       55.181        4.235       50.563      3.578       42.456        4.081       38.560      5.428   
## age         45.177      4.273       36.722        2.757       34.636      2.291       31.909        2.500       33.788      3.451   
## agesq       -0.781      0.076       -0.591        0.049       -0.530      0.041       -0.440        0.044       -0.436      0.062   
## hsgrad      28.580      7.314       25.078        4.769       17.543      3.824       15.024        4.389       13.355      5.868   
## somecoll    49.494      8.225       40.549        5.237       31.698      4.445       26.975        4.989       18.640      6.672   
## collgrad    82.645      8.799       57.601        5.766       36.839      4.816       16.270        5.422       -4.663      7.477   
## wtgain      11.776      0.167        9.905        0.119       9.134       0.103        8.838        0.118       8.576       0.157   
## natal2      6.176       6.555        2.833        4.171       -4.477      3.673       -0.544        4.075       3.796       5.633   
## natal3      39.434      15.115      12.679        9.600       3.967       7.234       -6.239        8.870      -25.498      10.738  
## novisit    -388.584     44.089     -196.452      24.286      -147.230     15.300     -126.642      16.132      -101.659     17.282  
## nosmoke    170.750      11.160      171.840       7.211      158.821      6.109       153.227       6.654      150.219      9.445   
## cigar       -3.803      0.640       -3.519        0.466       -3.902      0.390       -4.461        0.408       -5.168      0.607   
## boy         87.763      4.503       109.568       2.934      129.128      2.553       142.402       2.862      153.582      3.838   
## ------------------------------------------------------------------------------------------------------------------------------------
\end{verbatim}

As aforementioned in the case of a vector of 20 or 50 quantiles it is
better to use a \textbf{user written function}. We will use this
opportunity to remember how to define user written functions (even if in
this example is not necessary). An example of a function is presented to
highlight the important parts:

\begin{itemize}
\tightlist
\item
  Define the name
\item
  Define in parenteses the inputs and parameters of the function -
  remember that default values and ordering are important
\item
  Define in brackets the procedure of the function
\item
  Return the output, results of the operation
\end{itemize}

\begin{Shaded}
\begin{Highlighting}[]
\NormalTok{table_rq_beta_sd <-}\StringTok{ }\ControlFlowTok{function}\NormalTok{(qr_obj)\{}
        \CommentTok{# The function creates a summary table from the results of the command rq()}
        
\NormalTok{        len <-}\StringTok{ }\KeywordTok{length}\NormalTok{(qr_obj)}
        
\NormalTok{        res_beta <-}\StringTok{ }\KeywordTok{c}\NormalTok{()}
        \ControlFlowTok{for}\NormalTok{ (i }\ControlFlowTok{in} \DecValTok{1}\OperatorTok{:}\NormalTok{len) \{}
\NormalTok{                res <-}\StringTok{ }\NormalTok{qr_results[[i]]}\OperatorTok{$}\NormalTok{coefficients[,}\DecValTok{1}\NormalTok{] }
\NormalTok{                res_beta <-}\StringTok{ }\KeywordTok{cbind}\NormalTok{(res_beta,res)}
\NormalTok{        \}}
        
\NormalTok{        res_se <-}\StringTok{ }\KeywordTok{c}\NormalTok{()}
        \ControlFlowTok{for}\NormalTok{ (i }\ControlFlowTok{in} \DecValTok{1}\OperatorTok{:}\NormalTok{len) \{}
\NormalTok{                resse <-}\StringTok{ }\NormalTok{qr_results[[i]]}\OperatorTok{$}\NormalTok{coefficients[,}\DecValTok{2}\NormalTok{] }
\NormalTok{                res_se <-}\StringTok{ }\KeywordTok{cbind}\NormalTok{(res_se,resse)}
\NormalTok{        \}}
        
\NormalTok{        tau <-}\StringTok{ }\KeywordTok{c}\NormalTok{()}
        \ControlFlowTok{for}\NormalTok{ (i }\ControlFlowTok{in} \DecValTok{1}\OperatorTok{:}\NormalTok{len) \{}
\NormalTok{                tau <-}\StringTok{ }\KeywordTok{c}\NormalTok{(tau,qr_results[[i]]}\OperatorTok{$}\NormalTok{tau)}
\NormalTok{        \}}
        
\NormalTok{        res_beta <-}\StringTok{ }\KeywordTok{as.data.frame}\NormalTok{(res_beta)}
        \KeywordTok{names}\NormalTok{(res_beta) <-}\StringTok{ }\KeywordTok{paste0}\NormalTok{(}\StringTok{"Quant_"}\NormalTok{,tau,}\StringTok{"_beta"}\NormalTok{)}
\NormalTok{        res_se <-}\StringTok{ }\KeywordTok{as.data.frame}\NormalTok{(res_se)}
        \KeywordTok{names}\NormalTok{(res_se) <-}\StringTok{ }\KeywordTok{paste0}\NormalTok{(}\StringTok{"Quant_"}\NormalTok{,tau,}\StringTok{"_se"}\NormalTok{)}
        
        
\NormalTok{        results <-}\StringTok{ }\KeywordTok{cbind.data.frame}\NormalTok{(res_beta,res_se)}
\NormalTok{        results <-}\StringTok{ }\NormalTok{results[, }\KeywordTok{order}\NormalTok{(}\KeywordTok{names}\NormalTok{(results))]}
        \KeywordTok{return}\NormalTok{(results)}
\NormalTok{\}}
\end{Highlighting}
\end{Shaded}

Now we just have to call the function and introduce the object we
created before:

\begin{Shaded}
\begin{Highlighting}[]
\KeywordTok{table_rq_beta_sd}\NormalTok{(qr_results)}
\end{Highlighting}
\end{Shaded}

\begin{verbatim}
##             Quant_0.1_beta Quant_0.1_se Quant_0.25_beta Quant_0.25_se
## (Intercept)   1556.9050909  58.45879291    2020.4884992   37.98519460
## black         -253.6555790   7.91664698    -216.7938389    4.64081903
## married         73.3231343   6.55406020      55.1809157    4.23477730
## age             45.1774681   4.27347783      36.7216751    2.75655948
## agesq           -0.7810851   0.07629657      -0.5914186    0.04892666
## hsgrad          28.5797213   7.31354472      25.0779441    4.76923272
## somecoll        49.4943765   8.22498108      40.5488870    5.23705765
## collgrad        82.6454195   8.79916500      57.6005742    5.76601552
## wtgain          11.7755114   0.16691643       9.9051687    0.11880091
## natal2           6.1764400   6.55495922       2.8329988    4.17057029
## natal3          39.4337385  15.11503777      12.6792476    9.60040130
## novisit       -388.5838921  44.08936522    -196.4515370   24.28562273
## nosmoke        170.7501323  11.15950768     171.8397939    7.21094247
## cigar           -3.8030072   0.63954401      -3.5194985    0.46624584
## boy             87.7634154   4.50293347     109.5682422    2.93365299
##             Quant_0.5_beta Quant_0.5_se Quant_0.75_beta Quant_0.75_se
## (Intercept)   2377.8834596  31.58611134    2715.2279805   34.34313590
## black         -199.0709271   3.82873901    -192.4048070    4.26346977
## married         50.5633497   3.57758349      42.4560697    4.08113674
## age             34.6358786   2.29094506      31.9092314    2.50036832
## agesq           -0.5296273   0.04063692      -0.4395786    0.04427096
## hsgrad          17.5425396   3.82368815      15.0237215    4.38866496
## somecoll        31.6983508   4.44505051      26.9749700    4.98902204
## collgrad        36.8394529   4.81608945      16.2696096    5.42208252
## wtgain           9.1341375   0.10253622       8.8382909    0.11771601
## natal2          -4.4767476   3.67307972      -0.5437404    4.07482199
## natal3           3.9673501   7.23390304      -6.2389299    8.86964396
## novisit       -147.2300835  15.30004647    -126.6415003   16.13167721
## nosmoke        158.8209086   6.10852283     153.2272351    6.65373468
## cigar           -3.9020968   0.39009317      -4.4612046    0.40831223
## boy            129.1275630   2.55257411     142.4019151    2.86196534
##             Quant_0.9_beta Quant_0.9_se
## (Intercept)   2967.6827348  47.25148330
## black         -182.0865220   5.68366323
## married         38.5599376   5.42802626
## age             33.7876428   3.45079678
## agesq           -0.4355504   0.06158277
## hsgrad          13.3554587   5.86831170
## somecoll        18.6398317   6.67219287
## collgrad        -4.6634288   7.47727096
## wtgain           8.5759164   0.15652917
## natal2           3.7963741   5.63320458
## natal3         -25.4976502  10.73837729
## novisit       -101.6589089  17.28236339
## nosmoke        150.2194090   9.44533549
## cigar           -5.1678773   0.60667596
## boy            153.5822439   3.83777441
\end{verbatim}

If instead we want to reproduce the table of the paper, we need to
compute each of the QR in a separate object and calculate the OLS. Given
that the QR regression results table has a different variable order than
before, we use for comparison. We also redefine the formula to preserve
such order.

\begin{Shaded}
\begin{Highlighting}[]
\NormalTok{order_qr <-}\StringTok{ }\KeywordTok{c}\NormalTok{(}\StringTok{"birwt"}\NormalTok{,}\StringTok{"black"}\NormalTok{, }\StringTok{"married"}\NormalTok{, }\StringTok{"boy"}\NormalTok{, }\StringTok{"nosmoke"}\NormalTok{, }\StringTok{"cigar"}\NormalTok{, }\StringTok{"age"}\NormalTok{, }\StringTok{"agesq"}\NormalTok{, }\StringTok{"hsgrad"}\NormalTok{, }\StringTok{"somecoll"}\NormalTok{, }\StringTok{"collgrad"}\NormalTok{,}\StringTok{"wtgain"}\NormalTok{, }\StringTok{"natal2"}\NormalTok{, }\StringTok{"natal3"}\NormalTok{, }\StringTok{"novisit"}\NormalTok{)}

\NormalTok{data <-}\StringTok{ }\NormalTok{data[,order_qr]}
\NormalTok{Y <-}\StringTok{ "birwt"}
\NormalTok{X <-}\StringTok{ }\KeywordTok{paste}\NormalTok{(}\KeywordTok{names}\NormalTok{(data[,}\OperatorTok{-}\DecValTok{1}\NormalTok{]),}\DataTypeTok{collapse =} \StringTok{" + "}\NormalTok{)}
\NormalTok{formula_qr <-}\StringTok{ }\KeywordTok{as.formula}\NormalTok{(}\KeywordTok{paste}\NormalTok{(Y, }\StringTok{" ~ "}\NormalTok{, X))}
\NormalTok{formula_qr}
\end{Highlighting}
\end{Shaded}

\begin{verbatim}
## birwt ~ black + married + boy + nosmoke + cigar + age + agesq + 
##     hsgrad + somecoll + collgrad + wtgain + natal2 + natal3 + 
##     novisit
\end{verbatim}

\begin{Shaded}
\begin{Highlighting}[]
\NormalTok{p10 <-}\StringTok{ }\KeywordTok{rq}\NormalTok{(formula_qr,}\DataTypeTok{tau =} \KeywordTok{c}\NormalTok{(}\FloatTok{0.1}\NormalTok{), }\DataTypeTok{data =}\NormalTok{ data, }\DataTypeTok{method =} \StringTok{'fn'}\NormalTok{)}
\NormalTok{p25 <-}\StringTok{ }\KeywordTok{rq}\NormalTok{(formula_qr,}\DataTypeTok{tau =} \KeywordTok{c}\NormalTok{(}\FloatTok{0.25}\NormalTok{), }\DataTypeTok{data =}\NormalTok{ data, }\DataTypeTok{method =} \StringTok{'fn'}\NormalTok{)}
\NormalTok{p50 <-}\StringTok{ }\KeywordTok{rq}\NormalTok{(formula_qr,}\DataTypeTok{tau =} \KeywordTok{c}\NormalTok{(}\FloatTok{0.5}\NormalTok{), }\DataTypeTok{data =}\NormalTok{ data, }\DataTypeTok{method =} \StringTok{'fn'}\NormalTok{)}
\NormalTok{p75 <-}\StringTok{ }\KeywordTok{rq}\NormalTok{(formula_qr,}\DataTypeTok{tau =} \KeywordTok{c}\NormalTok{(}\FloatTok{0.75}\NormalTok{), }\DataTypeTok{data =}\NormalTok{ data, }\DataTypeTok{method =} \StringTok{'fn'}\NormalTok{)}
\NormalTok{p90 <-}\StringTok{ }\KeywordTok{rq}\NormalTok{(formula_qr,}\DataTypeTok{tau =} \KeywordTok{c}\NormalTok{(}\FloatTok{0.9}\NormalTok{), }\DataTypeTok{data =}\NormalTok{ data, }\DataTypeTok{method =} \StringTok{'fn'}\NormalTok{)}
\NormalTok{ols <-}\StringTok{ }\KeywordTok{lm}\NormalTok{(formula_qr, }\DataTypeTok{data =}\NormalTok{ data)}
\end{Highlighting}
\end{Shaded}

Finally, we put the results in a paper format using LaTeX syntax and
`stargazer' functionality. The table presented shows the results for the
QR estimation.

\begin{Shaded}
\begin{Highlighting}[]
\KeywordTok{stargazer}\NormalTok{(p10,p25,p50,p75,p90,ols, }\DataTypeTok{title =} \StringTok{"Quantile Regression Results"}\NormalTok{, }
          \DataTypeTok{type=}\KeywordTok{ifelse}\NormalTok{(knitr}\OperatorTok{::}\KeywordTok{is_latex_output}\NormalTok{(),}\StringTok{"latex"}\NormalTok{,}\StringTok{"html"}\NormalTok{), }\DataTypeTok{out =}\StringTok{"./images/qr_rep_tab_92.tex"}\NormalTok{)}
\end{Highlighting}
\end{Shaded}

\% Table created by stargazer v.5.2.2 by Marek Hlavac, Harvard
University. E-mail: hlavac at fas.harvard.edu \% Date and time: Mon, Oct
07, 2019 - 19:27:44

\begin{table}[!htbp] \centering 
  \caption{Quantile Regression Results} 
  \label{tab:} 
\begin{tabular}{@{\extracolsep{5pt}}lcccccc} 
\\[-1.8ex]\hline 
\hline \\[-1.8ex] 
 & \multicolumn{6}{c}{\textit{Dependent variable:}} \\ 
\cline{2-7} 
\\[-1.8ex] & \multicolumn{6}{c}{birwt} \\ 
\\[-1.8ex] & \multicolumn{5}{c}{\textit{quantile}} & \textit{OLS} \\ 
 & \multicolumn{5}{c}{\textit{regression}} & \textit{} \\ 
\\[-1.8ex] & (1) & (2) & (3) & (4) & (5) & (6)\\ 
\hline \\[-1.8ex] 
 black & $-$253.656$^{***}$ & $-$216.794$^{***}$ & $-$199.071$^{***}$ & $-$192.405$^{***}$ & $-$182.087$^{***}$ & $-$220.271$^{***}$ \\ 
  & (7.917) & (4.641) & (3.829) & (4.263) & (5.684) & (3.574) \\ 
  & & & & & & \\ 
 married & 73.323$^{***}$ & 55.181$^{***}$ & 50.563$^{***}$ & 42.456$^{***}$ & 38.560$^{***}$ & 57.560$^{***}$ \\ 
  & (6.554) & (4.235) & (3.578) & (4.081) & (5.428) & (3.337) \\ 
  & & & & & & \\ 
 boy & 87.763$^{***}$ & 109.568$^{***}$ & 129.128$^{***}$ & 142.402$^{***}$ & 153.582$^{***}$ & 122.568$^{***}$ \\ 
  & (4.503) & (2.934) & (2.553) & (2.862) & (3.838) & (2.385) \\ 
  & & & & & & \\ 
 nosmoke & 170.750$^{***}$ & 171.840$^{***}$ & 158.821$^{***}$ & 153.227$^{***}$ & 150.219$^{***}$ & 161.929$^{***}$ \\ 
  & (11.160) & (7.211) & (6.109) & (6.654) & (9.445) & (5.643) \\ 
  & & & & & & \\ 
 cigar & $-$3.803$^{***}$ & $-$3.519$^{***}$ & $-$3.902$^{***}$ & $-$4.461$^{***}$ & $-$5.168$^{***}$ & $-$4.043$^{***}$ \\ 
  & (0.640) & (0.466) & (0.390) & (0.408) & (0.607) & (0.367) \\ 
  & & & & & & \\ 
 age & 45.177$^{***}$ & 36.722$^{***}$ & 34.636$^{***}$ & 31.909$^{***}$ & 33.788$^{***}$ & 37.584$^{***}$ \\ 
  & (4.273) & (2.757) & (2.291) & (2.500) & (3.451) & (2.071) \\ 
  & & & & & & \\ 
 agesq & $-$0.781$^{***}$ & $-$0.591$^{***}$ & $-$0.530$^{***}$ & $-$0.440$^{***}$ & $-$0.436$^{***}$ & $-$0.579$^{***}$ \\ 
  & (0.076) & (0.049) & (0.041) & (0.044) & (0.062) & (0.036) \\ 
  & & & & & & \\ 
 hsgrad & 28.580$^{***}$ & 25.078$^{***}$ & 17.543$^{***}$ & 15.024$^{***}$ & 13.355$^{**}$ & 16.684$^{***}$ \\ 
  & (7.314) & (4.769) & (3.824) & (4.389) & (5.868) & (3.653) \\ 
  & & & & & & \\ 
 somecoll & 49.494$^{***}$ & 40.549$^{***}$ & 31.698$^{***}$ & 26.975$^{***}$ & 18.640$^{***}$ & 30.372$^{***}$ \\ 
  & (8.225) & (5.237) & (4.445) & (4.989) & (6.672) & (4.180) \\ 
  & & & & & & \\ 
 collgrad & 82.645$^{***}$ & 57.601$^{***}$ & 36.839$^{***}$ & 16.270$^{***}$ & $-$4.663 & 36.718$^{***}$ \\ 
  & (8.799) & (5.766) & (4.816) & (5.422) & (7.477) & (4.580) \\ 
  & & & & & & \\ 
 wtgain & 11.776$^{***}$ & 9.905$^{***}$ & 9.134$^{***}$ & 8.838$^{***}$ & 8.576$^{***}$ & 10.490$^{***}$ \\ 
  & (0.167) & (0.119) & (0.103) & (0.118) & (0.157) & (0.098) \\ 
  & & & & & & \\ 
 natal2 & 6.176 & 2.833 & $-$4.477 & $-$0.544 & 3.796 & 7.731$^{**}$ \\ 
  & (6.555) & (4.171) & (3.673) & (4.075) & (5.633) & (3.435) \\ 
  & & & & & & \\ 
 natal3 & 39.434$^{***}$ & 12.679 & 3.967 & $-$6.239 & $-$25.498$^{**}$ & 20.268$^{***}$ \\ 
  & (15.115) & (9.600) & (7.234) & (8.870) & (10.738) & (7.564) \\ 
  & & & & & & \\ 
 novisit & $-$388.584$^{***}$ & $-$196.452$^{***}$ & $-$147.230$^{***}$ & $-$126.642$^{***}$ & $-$101.659$^{***}$ & $-$193.737$^{***}$ \\ 
  & (44.089) & (24.286) & (15.300) & (16.132) & (17.282) & (12.534) \\ 
  & & & & & & \\ 
 Constant & 1,556.905$^{***}$ & 2,020.488$^{***}$ & 2,377.883$^{***}$ & 2,715.228$^{***}$ & 2,967.683$^{***}$ & 2,273.476$^{***}$ \\ 
  & (58.459) & (37.985) & (31.586) & (34.343) & (47.251) & (28.700) \\ 
  & & & & & & \\ 
\hline \\[-1.8ex] 
Observations & 199,181 & 199,181 & 199,181 & 199,181 & 199,181 & 199,181 \\ 
R$^{2}$ &  &  &  &  &  & 0.134 \\ 
Adjusted R$^{2}$ &  &  &  &  &  & 0.134 \\ 
Residual Std. Error &  &  &  &  &  & 531.845 (df = 199166) \\ 
F Statistic &  &  &  &  &  & 2,202.321$^{***}$ (df = 14; 199166) \\ 
\hline 
\hline \\[-1.8ex] 
\textit{Note:}  & \multicolumn{6}{r}{$^{*}$p$<$0.1; $^{**}$p$<$0.05; $^{***}$p$<$0.01} \\ 
\end{tabular} 
\end{table}

\subsection{Quantile Regression
visualization}\label{quantile-regression-visualization}

One of the most important ways to visualize and communicate the results
from QR is plots. The `quantreg' package has its own functionality. You
will obtain graphs with or without confidence depending on the object
you feed it: without CI if using the qr object before the summary, and
with CI if it is fed the summary of a QR object. To plot we use the
command \texttt{plot()}. In the first case we obtain all graphs for the
previous estimation. A similar graph is presented in
\citep{koenker2001quantile}, in which the Authors the method and apply
it to 1997 data (at this point you can reproduce the plots by yourself).

\begin{Shaded}
\begin{Highlighting}[]
\KeywordTok{plot}\NormalTok{(quant_reg_res1992)}
\end{Highlighting}
\end{Shaded}

\includegraphics{Material_files/figure-latex/unnamed-chunk-24-1.pdf}

If instead we want to inspect the effect we need more points, so more
quantiles. We construct 19 observations and used the CI from the
bootstrap option (as stated on the paper). In this case we only show the
second independent variable (black dummy).

\begin{Shaded}
\begin{Highlighting}[]
\NormalTok{time_}\DecValTok{0}\NormalTok{ <-}\StringTok{ }\KeywordTok{Sys.time}\NormalTok{()}
\NormalTok{reg_exp <-}\StringTok{ }\KeywordTok{rq}\NormalTok{(formula_qr,}\DataTypeTok{tau =} \KeywordTok{seq}\NormalTok{(}\FloatTok{0.05}\NormalTok{,}\FloatTok{0.95}\NormalTok{,}\DataTypeTok{by =} \FloatTok{0.05}\NormalTok{), }\DataTypeTok{data =}\NormalTok{ data, }\DataTypeTok{method =} \StringTok{'fn'}\NormalTok{)}
\NormalTok{time_}\DecValTok{1}\NormalTok{ <-}\StringTok{ }\KeywordTok{Sys.time}\NormalTok{()}

\KeywordTok{paste0}\NormalTok{(}\StringTok{"Computing the quantiles took "}\NormalTok{,time_}\DecValTok{1}\OperatorTok{-}\NormalTok{time_}\DecValTok{0}\NormalTok{)}
\end{Highlighting}
\end{Shaded}

\begin{verbatim}
## [1] "Computing the quantiles took 30.3356010913849"
\end{verbatim}

\begin{Shaded}
\begin{Highlighting}[]
\NormalTok{sum_reg <-}\StringTok{ }\KeywordTok{summary.rqs}\NormalTok{(reg_exp, }\DataTypeTok{method =} \StringTok{'boot'}\NormalTok{)}
\end{Highlighting}
\end{Shaded}

\begin{verbatim}
## Warning in summary.rq(xi, U = U, ...): 98 non-positive fis
\end{verbatim}

\begin{Shaded}
\begin{Highlighting}[]
\NormalTok{time_}\DecValTok{2}\NormalTok{ <-}\StringTok{ }\KeywordTok{Sys.time}\NormalTok{()}

\KeywordTok{paste0}\NormalTok{(}\StringTok{"Computing the errors took "}\NormalTok{,time_}\DecValTok{2}\OperatorTok{-}\NormalTok{time_}\DecValTok{1}\NormalTok{)}
\end{Highlighting}
\end{Shaded}

\begin{verbatim}
## [1] "Computing the errors took 1.09833411375682"
\end{verbatim}

\begin{Shaded}
\begin{Highlighting}[]
\KeywordTok{plot}\NormalTok{(sum_reg, }\DataTypeTok{parm =} \KeywordTok{c}\NormalTok{(}\DecValTok{2}\NormalTok{))}
\end{Highlighting}
\end{Shaded}

\includegraphics{Material_files/figure-latex/unnamed-chunk-25-1.pdf}

The plot shows the estimates values, the confidence intervals and the
OLS estimated value, so we can compare if the QR offers additional
insights.

\section{Exercises}\label{exercises}

\subsection{Proposed exercise 1}\label{proposed-exercise-1}

\begin{itemize}
\tightlist
\item
  Compute the quantile regression for the year 1996 to complete the
  descriptive statistics of the paper's table. Are they similar to 1992
  results? What is equal? What is different?
\item
  Compute the quantile regression for the same set of variables but for
  a recent year (after year 2000). Does the outcome change? If yes, what
  changes and how would you interpret it?
\end{itemize}

\subsection{Proposed exercise 2}\label{proposed-exercise-2}

\begin{itemize}
\tightlist
\item
  Compute the quantile regression for the year 1992 and 1996 including a
  variable of your interest. Does the result change significantly? Why
  do you consider it relevant for the analysis? How do you interpret the
  results obtained?
\end{itemize}

\section{More on quantiles}\label{more-on-quantiles}

In this section I will only list other implementations of quantile
regression that might be usefull for your future research. The list is
presentend without any particular order. If you have any suggestions,
please let me know:

\begin{itemize}
\tightlist
\item
  Decomposition methods using quantiles
  \citep{machado2005counterfactual}
\item
  Un-conditional quantiles regression \citep{firpo2009unconditional}
\item
  Decomposition methods using un-conditional quantile methods
  \citep{firpo2018decomposing}
\item
  Quantile estimation with non linear effects
\item
  Parallel quantile estimation
\item
  Quantile regression for time series (CAViaR)
\item
  Quantile regression for Spatial Data (package McSpatial)
\item
  Quantile regression for panel data, which is under development since
  it does not exist (yet) a consistent estimator (package `rqpd' and
  Ivan Canay's package)
\end{itemize}

\section{References}\label{references}

This material was possible thanks to the slides of David MARGOLIS (in
PSE resources), the
\href{https://freakonometrics.hypotheses.org/files/2017/05/erasmus-1.pdf}{slides
of Arthur CHARPENTIER}, and the
\href{http://www.crest.fr/ckfinder/userfiles/files/Pageperso/xdhaultfoeuille/course_qreg.pdf}{slides
of of the course of Xavier D'HAULTFOEUILLE} (ENSAE).

\chapter{Session II - Maximum Likelihood Estimation
(MLE)}\label{session-ii---maximum-likelihood-estimation-mle}

\begin{center}\includegraphics[width=0.9\linewidth]{./images/1200px-Logo_pse_petit} \end{center}

\section{Introduction}\label{introduction-1}

In this second session of the microeconometrics tutorial we are going to
implement Maximum Likelihood Estimation in R. The essential steps are:

\begin{enumerate}
\def\labelenumi{\arabic{enumi}.}
\item
  Understand the intuition behind Maximum likelihood estimation. We
  awill replicate a Poisson regression table using MLE. We will both
  write our own custom function and a built-in one.
\item
  Propose a model and derive its likelihood function. This part is not
  going to be very deep in the explanation of the model, derivation and
  assumptions. The aim of this sessions are on the estimation
  (computation) and not in the model per se. I will however refer the
  sources if you want to have a deeper look at the model. Given that the
  last session we did something on health economics, this time we change
  topic and will focus on labour economics. As anecdote, when I first
  saw this I was very impressed! I hope you are impressed too after
  today session!
\item
  Use the Current Population Survey (CPS) and understand how to handle
  and manage this particular dataset.

\begin{verbatim}
+ Import the data
+ Use the specified columns
+ Clean data
\end{verbatim}
\item
  Estimate the structural parameters of the proposed model (both for the
  estimates and the standard errors, obtaind via the \emph{delta
  method}).
\end{enumerate}

\section{Maximum likelihood
estimator}\label{maximum-likelihood-estimator}

For the theory please refer to the slides of the course (class 2). Here
we just provide a brief definition and the intuition to the method
application. What is Maximum likelihood estimation (MLE)? MLE is an
estimation method in which we obtain the \emph{parameters} of our model
under an \textbf{assumed} \emph{statistical model} and the available
\emph{data}, such that our sample is the most probable.

\begin{itemize}
\item
  Given a statistical model (ie, an economic model with suitable
  stochastic features), select the parameters that make the observed
  data most probable. In this way, we are doing inference in the
  population that generated our data and the DGP behind. We can
  formulate \textbf{any} model and we will obtain a result; the only
  restriction for the formulation is that it has probability 0.
\item
  Even if it is intuitive, rely on the \textbf{assumptions} (model,
  statistical model, DGP), but not in the \textbf{validity}.
\item
  Validity of the models?
\end{itemize}

\begin{quote}
``A model is a deliberate abstraction from reality. One model can be
better than another, on one or several dimensions, but none are correct.
They help us focus on the small set of phenomena in which we are
interested, and/or have data regarding. When correctly developed and
explained, it should be clear what set of phenomena are being excluded
from consideration, and, at the end of the analysis, it is desirable to
say how the omission of other relevant phenomena could have affected the
results attained.
\end{quote}

\begin{quote}
An advantage of a structured approach to empirical analysis is that it
should be immediately clear what factors have been considered exogenous
and which endogenous, the functional form assumptions made, etc." (C.
Flinn,
\href{http://www.econ.nyu.edu/user/flinnc/courses/CCA-RGSE-2019/search_estimation_notes.pdf}{lecture
notes})
\end{quote}

To understand how MLE works we will use two examples today: a Poisson
regression and a structural estimation.

\subsection{Poisson regression}\label{poisson-regression}

In this section we are going to replicate a paper by
\href{https://www.danieltreisman.org/}{Daniel Treisman} published in AER
2016. It applies Poisson regression to the number of millioners in a set
of countries, conditional on some country characteristics
\citep{treisman2016russia}.

\begin{itemize}
\tightlist
\item
  \textbf{What does the paper says?}
\end{itemize}

The main idea of the paper is \emph{provide} a robust \emph{model} to
predict the number of rich in a given country, given an economic
environment specific to said country.

In comparing data and predictions the Author finds that, regardless of
the model specification, Russia reports the highest number of anomalies
(underpredictions).

\begin{itemize}
\tightlist
\item
  \textbf{Why a Poisson regression?}
\end{itemize}

In Treisman's paper the dependent variable --- the number of
billionaires \(y_i\) in country \(i\) --- is modelled as a function of
GDP per capita, population size, and years membership in GATT and WTO.
He also presents 4 alternative specifications.

\begin{quote}
``\ldots{} since the dependent variable is a count, Poisson rather than
OLS regression is appropriate.''
\end{quote}

\subsubsection{Estimation}\label{estimation}

You can acces the
\href{https://www.aeaweb.org/aer/data/10605/P2016_1068_data.zip}{data}
and the
\href{https://pubs.aeaweb.org/doi/pdfplus/10.1257/aer.p20161068}{paper}
in the provided links. Download and unzip the information. The file also
contains a companion STATA code to reproduce the tables in the paper.
Unzip the information and load the dataset in R using the \texttt{haven}
library \citep{R-haven}:

\begin{Shaded}
\begin{Highlighting}[]
\NormalTok{data_mil <-}\StringTok{ }\KeywordTok{read_dta}\NormalTok{(}\StringTok{"YOUR PATH GOES HERE"}\NormalTok{)}
\end{Highlighting}
\end{Shaded}

To reproduce table 1 from the paper we need to filter the information
for 2008, as it is the year considered in the main analisys.

\begin{Shaded}
\begin{Highlighting}[]
\NormalTok{data_mil  <-}\StringTok{ }\NormalTok{data_mil[data_mil}\OperatorTok{$}\NormalTok{year }\OperatorTok{==}\StringTok{ }\DecValTok{2008}\NormalTok{,]}
\end{Highlighting}
\end{Shaded}

Our \textbf{goal} is to \textbf{estimate a Poisson regression model} and
there are built-in functions to do these kind of estimations using a
one-line command like \texttt{glm(...,\ family\ =\ "poisson")}. Our
\textbf{goal} instead is to \textbf{use Maximum Likelihood estimation to
reproduce such parameters} and understand how this works. In order to
have a benchmark for comparison let's see how the output of the first
proposed model looks like:

\begin{Shaded}
\begin{Highlighting}[]
\KeywordTok{summary}\NormalTok{(}\KeywordTok{glm}\NormalTok{(}\DataTypeTok{formula =}\NormalTok{ numbil0 }\OperatorTok{~}\StringTok{ }\NormalTok{lngdppc }\OperatorTok{+}\StringTok{ }\NormalTok{lnpop }\OperatorTok{+}\StringTok{ }\NormalTok{gattwto08, }\DataTypeTok{family =} \KeywordTok{poisson}\NormalTok{(}\DataTypeTok{link=}\StringTok{'log'}\NormalTok{), }\DataTypeTok{data =}\NormalTok{ data_mil))}
\end{Highlighting}
\end{Shaded}

\begin{verbatim}
## 
## Call:
## glm(formula = numbil0 ~ lngdppc + lnpop + gattwto08, family = poisson(link = "log"), 
##     data = data_mil)
## 
## Deviance Residuals: 
##     Min       1Q   Median       3Q      Max  
## -8.7615  -0.7585  -0.3775  -0.1010   9.0587  
## 
## Coefficients:
##               Estimate Std. Error z value Pr(>|z|)    
## (Intercept) -29.049536   0.638193 -45.518  < 2e-16 ***
## lngdppc       1.083856   0.035064  30.911  < 2e-16 ***
## lnpop         1.171362   0.024156  48.491  < 2e-16 ***
## gattwto08     0.005968   0.001908   3.127  0.00176 ** 
## ---
## Signif. codes:  0 '***' 0.001 '**' 0.01 '*' 0.05 '.' 0.1 ' ' 1
## 
## (Dispersion parameter for poisson family taken to be 1)
## 
##     Null deviance: 5942.23  on 196  degrees of freedom
## Residual deviance:  669.95  on 193  degrees of freedom
##   (354 observations deleted due to missingness)
## AIC: 885.08
## 
## Number of Fisher Scoring iterations: 5
\end{verbatim}

Let's start the construction of the maximum likelihood introducing the
Poisson regression model. The starting assumptions are: Poisson
regression is a generalized linear model form of regression analysis
used to model count data \((1)\), in which the dependent variable has a
Poisson distribution \((2)\), and the logarithm of the expected value
can be modeled as a linear combination \((3)\).

\begin{enumerate}
\def\labelenumi{\arabic{enumi}.}
\tightlist
\item
  Count data: notice that all the numbers from a Poisson distribution
  are integers.
\item
  Poisson pdf is defined as:
  \[f(k,\lambda)=Pr(X=k)=\frac{\lambda^k e^{-\lambda}}{k!}\] Keep in
  mind that the \textbf{mean} and variance of the Posson distribution
  are equal to the constant \(\lambda\).
\item
  The \textbf{mean} can be expressed as a linear combination of the
  parameters.
  \[ \mu = E[y] = E[e^{y}] = E(y| \boldsymbol{X})=e^{\theta^{\prime} \boldsymbol{X}}=e^{\theta_0 + \theta_1 x_{i1}+ ... + \theta_k x_{ik} }\]
  Combining condition \((2)\) and \((3)\), we obtain the probability
  density function:
\end{enumerate}

\[f(y_i,\lambda)=f(y_i,\mu)=\frac{\mu^y_i e^{-\mu}}{y_i!}\]

The joint probability density function is hence equal to:

\[f(\boldsymbol{Y}|\boldsymbol{X},\boldsymbol{\theta})=f(y_1,\mu) f(y_2,\mu)...f(y_n,\mu)    = \prod_{i=1}^n f(y_i,\mu|\boldsymbol{X},\boldsymbol{\theta})\]

Our likelihood function instead takes as given the data (vector
\(\boldsymbol{Y}\) and matrix \(\boldsymbol{X}\)) and calculates the
likelihood given a parameter \(\theta\).

\[\mathcal{L}(\boldsymbol{\theta} | \boldsymbol{Y},\boldsymbol{X}) = \prod_{i=1}^n f(\boldsymbol{\theta}|\boldsymbol{X},\boldsymbol{Y}) = \prod_{i=1}^n \frac{\mu_i^{y_i} e^{-\mu_i}}{y_i!}\]

Finding the \(\boldsymbol{\hat{\theta}}\) that maximizes the likelihood
function therefore boils down to estimate the model parameters. Before
this step we monotically transform the obective function to feed the
algorithm the log-likelihood instead of the function itself. Why? Think
about derivatives of sums vs.~derivatives of products.
\[\boldsymbol{\hat{\theta}} = \max_{\boldsymbol{\theta}} log( \mathcal{L}(\boldsymbol{\theta} | \boldsymbol{Y},\boldsymbol{X})) = \min _{\boldsymbol{\theta}} - log( \mathcal{L}(\boldsymbol{\theta} | \boldsymbol{Y},\boldsymbol{X})) \]
Taking the logarithm then we have:

\begin{equation}
\begin{aligned}
 \log( \mathcal{L}(\boldsymbol{\theta} | \boldsymbol{Y},\boldsymbol{X}))  &= \log \left( \prod_{i=1}^N \frac{\mu_i^y e^{-\mu}}{y_i!} \right) \\
 &= \sum_{i=1}^N \log \left( \frac{\mu_i^{y_i} e^{-\mu_i}}{y_i!} \right)\\
 &= \sum_{i=1}^N y_i \log \left( \mu_i \right) - \sum_{i=1}^N \left(\mu_i\right) -\sum_{i=1}^N \log \left(y_i! \right)\\
\end{aligned}
\end{equation}

Where

\[ \mu = e^{\theta^{\prime} \boldsymbol{X}}=e^{\theta_0 + \theta_1 x_{i1}+ ... + \theta_k x_{ik} }\].

Absent a closed-form solution, we are going to use the R optimizer to
maximize the log-likelihood function above. The \emph{optim} function
will serve this purpose. According to the help vignette \texttt{?optim},
we need:

\begin{itemize}
\item
  A vector of initial values to start the search from.
\item
  A well specified \textbf{function} to \textbf{minimize}. It must take
  as \emph{input} \textbf{the vector of parameters} and \emph{should}
  \textbf{return a scalar}.
\item
  \emph{Method} of optimization (?).
\item
  Optional: a hessian (boolean). We will use this to parse out the
  standard errors around the estimated parameters, so it will be useful
  later on.
\end{itemize}

In the last installment we introduced the basics of custom functions in
R. In this tutorial we just recall as good practice that we are going to
differenciate the inputs of the function: the parameters are the inputs
that are going to change in the optimization process, while the data for
example will be a \textbf{static} input. Another static input is the
\emph{formula}, a type of R object that will allow to extract the
relevant information such as variables name, data columns, and instructs
the algorithm on the relationship between variables (dependent or
independent, interactions, dependance, etc).

We start by writing the function. We call the function \emph{`LLcalc'},
and define the inputs in parentheses. One useful function that we use in
the first line is \texttt{model.matrix()}. This function takes a formula
and extract from the whole dataset the related matrix of observations
including the vector of ones of the intercept, dummies, and interaction
terms.

We are going to compute the value of \(\mu\) and then the value of the
individual contribution to the log-likelihood. Then, we just sum and
flip sign, as the optimizer minimizes by default. There is one technical
detail that we are addressing using the package \texttt{Rmpfr}, which
allow to store big numbers in 128 bits (as the factorail of 400)
\citep{R-Rmpfr}.

\begin{Shaded}
\begin{Highlighting}[]
\NormalTok{LLcalc <-}\StringTok{ }\ControlFlowTok{function}\NormalTok{(theta, formula, data)\{}
\NormalTok{        ### Calculate the log-likelihood of a Poisson regression, }
\NormalTok{        ### given a vector of parameters (1), }
\NormalTok{        ### a relationship ()formula (2) }
\NormalTok{        ### and a dataset (type: dataframe)}
        
\NormalTok{        rhs <-}\StringTok{ }\KeywordTok{model.matrix}\NormalTok{(formula, }\DataTypeTok{data =}\NormalTok{ data)}
        \KeywordTok{colnames}\NormalTok{(rhs)[}\DecValTok{1}\NormalTok{] <-}\StringTok{ "Intercept"}
\NormalTok{        Y <-}\StringTok{ }\KeywordTok{as.matrix}\NormalTok{(data[}\KeywordTok{rownames}\NormalTok{(rhs),}\KeywordTok{toString}\NormalTok{(formula[[}\DecValTok{2}\NormalTok{]])])}
        \CommentTok{# Expected values \textbackslash{}mu }
\NormalTok{        mu <-}\StringTok{ }\KeywordTok{exp}\NormalTok{(rhs }\OperatorTok\StringTok{ }\NormalTok{theta)}
        \CommentTok{# Value of the log likelihood}
\NormalTok{        LL <-}\StringTok{ }\NormalTok{Y }\OperatorTok{*}\StringTok{ }\KeywordTok{log}\NormalTok{(mu)}\OperatorTok{-}\NormalTok{mu}\OperatorTok{-}\KeywordTok{as.numeric}\NormalTok{(}\KeywordTok{log}\NormalTok{(}\KeywordTok{gamma}\NormalTok{(}\KeywordTok{as}\NormalTok{(Y}\OperatorTok{+}\DecValTok{1}\NormalTok{,}\StringTok{"mpfr"}\NormalTok{))))}
        
        \CommentTok{#print(cbind(colnames(rhs),round(theta,3)))}
        \KeywordTok{return}\NormalTok{(}\OperatorTok{-}\KeywordTok{sum}\NormalTok{(LL, }\DataTypeTok{na.rm =}\NormalTok{ T))}
        
\NormalTok{\}}
\end{Highlighting}
\end{Shaded}

To check whether the function works properly we feed it the data, a
random \(\theta\), and the formula of the first column of table I in the
paper.

\begin{Shaded}
\begin{Highlighting}[]
\NormalTok{theta_test <-}\StringTok{ }\KeywordTok{rnorm}\NormalTok{(}\DecValTok{4}\NormalTok{)}
\NormalTok{formula_}\DecValTok{1}\NormalTok{ <-}\StringTok{ }\KeywordTok{as.formula}\NormalTok{(numbil0 }\OperatorTok{~}\StringTok{ }\NormalTok{lngdppc }\OperatorTok{+}\StringTok{ }\NormalTok{lnpop }\OperatorTok{+}\StringTok{ }\NormalTok{gattwto08)}
\KeywordTok{LLcalc}\NormalTok{(}\DataTypeTok{theta =}\NormalTok{ theta_test, }\DataTypeTok{formula =}\NormalTok{ formula_}\DecValTok{1}\NormalTok{, }\DataTypeTok{data =}\NormalTok{ data_mil)}
\end{Highlighting}
\end{Shaded}

\begin{verbatim}
## [1] 26030.13
\end{verbatim}

Now that we know is working, it is time to set up the optimizer. As
discussed before we need the function, some starting parameters, the
data and the formula, the static inputs of our function.

\begin{Shaded}
\begin{Highlighting}[]
\NormalTok{theta_}\DecValTok{0}\NormalTok{ <-}\StringTok{ }\KeywordTok{c}\NormalTok{(}\OperatorTok{-}\DecValTok{30}\NormalTok{,}\DecValTok{1}\NormalTok{,}\DecValTok{1}\NormalTok{,}\DecValTok{0}\NormalTok{)}
\NormalTok{dTbp_beta <-}\StringTok{ }\KeywordTok{optim}\NormalTok{(}\DataTypeTok{par =}\NormalTok{ theta_}\DecValTok{0}\NormalTok{, }\DataTypeTok{fn =}\NormalTok{ LLcalc, }\DataTypeTok{data =}\NormalTok{ data_mil, }\DataTypeTok{formula =}\NormalTok{ formula_}\DecValTok{1}\NormalTok{, }\DataTypeTok{method =} \StringTok{"Nelder-Mead"}\NormalTok{, }\DataTypeTok{hessian =} \OtherTok{TRUE}\NormalTok{)}
\KeywordTok{round}\NormalTok{(dTbp_beta}\OperatorTok{$}\NormalTok{par,}\DecValTok{3}\NormalTok{)}
\end{Highlighting}
\end{Shaded}

\begin{verbatim}
## [1] -29.052   1.084   1.171   0.006
\end{verbatim}

To carry out the estimation we need to compute the standard errors. As a
by-product, we also have the Hessian, this is useful in couple with the
theory from the main class:

\begin{itemize}
\tightlist
\item
  The negative of the hessian (provided as a linearization around the
  optimum of the LL maximization) is equal to the Fisher information
  matrix.
\end{itemize}

\[[\mathcal{I}(\theta)]_{i,j}= \mathbf{E}\left[\left(\frac{\partial}{\partial \theta_i} \log f(X|\theta)\right)\left(\frac{\partial}{\partial \theta_j} \log f(X|\theta)\right)|\theta\right]\approx\left[\frac{\partial^2}{\partial \theta_i \partial \theta_j} \log f(X|\theta)|\theta\right]\]

\begin{itemize}
\item
  The Fisher information matrix, when inverted, is equal to the variance
  covariance matrix. (Formally, Cramer-Rao state that the inverse is the
  lower bound of the variance if the estimator is unbiased.)
\item
  The variance covariance matrix has in the diagonal the variane for
  each parameter. Taking square root of it gives the standard errors.
\item
  since we minimize we do not have to flip sign, as the values of the LL
  are calculated over the negative likelihood function.
\end{itemize}

Considering these notions we obtain the following standard errors that
are equal to the previous regression.

\begin{Shaded}
\begin{Highlighting}[]
\NormalTok{fisher_info <-}\StringTok{ }\NormalTok{dTbp_beta}\OperatorTok{$}\NormalTok{hessian}
\NormalTok{vcov <-}\StringTok{ }\KeywordTok{solve}\NormalTok{(fisher_info)}
\NormalTok{se <-}\StringTok{ }\KeywordTok{sqrt}\NormalTok{(}\KeywordTok{diag}\NormalTok{(vcov))}
\NormalTok{se}
\end{Highlighting}
\end{Shaded}

\begin{verbatim}
## [1] 0.638233253 0.035068731 0.024155062 0.001907374
\end{verbatim}

To complete the exercise we are going to reproduce the whole table 1
from the paper. We need to set up all of the formulas (models)
estimated. The paper provides the robust errors, so we calculate them
and format with stargazer \citep{R-stargazer} output.

\begin{Shaded}
\begin{Highlighting}[]
\NormalTok{formula_}\DecValTok{2}\NormalTok{ <-}\StringTok{ }\KeywordTok{as.formula}\NormalTok{(numbil0 }\OperatorTok{~}\StringTok{  }\NormalTok{lngdppc }\OperatorTok{+}\StringTok{ }\NormalTok{lnpop }\OperatorTok{+}\StringTok{  }\NormalTok{gattwto08 }\OperatorTok{+}\StringTok{  }\NormalTok{lnmcap08 }\OperatorTok{+}\StringTok{  }\NormalTok{rintr }\OperatorTok{+}\StringTok{  }\NormalTok{topint08)}
\NormalTok{formula_}\DecValTok{3}\NormalTok{ <-}\StringTok{ }\KeywordTok{as.formula}\NormalTok{(numbil0 }\OperatorTok{~}\StringTok{  }\NormalTok{lngdppc }\OperatorTok{+}\StringTok{ }\NormalTok{lnpop }\OperatorTok{+}\StringTok{  }\NormalTok{gattwto08 }\OperatorTok{+}\StringTok{  }\NormalTok{lnmcap08 }\OperatorTok{+}\StringTok{  }\NormalTok{rintr }\OperatorTok{+}\StringTok{  }\NormalTok{topint08 }\OperatorTok{+}\StringTok{  }\NormalTok{nrrents }\OperatorTok{+}\StringTok{  }\NormalTok{roflaw)}
\NormalTok{formula_}\DecValTok{4}\NormalTok{ <-}\StringTok{ }\KeywordTok{as.formula}\NormalTok{(numbil0 }\OperatorTok{~}\StringTok{  }\NormalTok{lngdppc }\OperatorTok{+}\StringTok{ }\NormalTok{lnpop }\OperatorTok{+}\StringTok{  }\NormalTok{gattwto08 }\OperatorTok{+}\StringTok{  }\NormalTok{lnmcap08 }\OperatorTok{+}\StringTok{  }\NormalTok{rintr }\OperatorTok{+}\StringTok{  }\NormalTok{topint08 }\OperatorTok{+}\StringTok{  }\NormalTok{nrrents }\OperatorTok{+}\StringTok{  }\NormalTok{roflaw }\OperatorTok{+}\StringTok{  }\NormalTok{fullprivproc)}

\NormalTok{r1 <-}\StringTok{ }\KeywordTok{glm}\NormalTok{(}\DataTypeTok{formula =}\NormalTok{ formula_}\DecValTok{1}\NormalTok{, }\DataTypeTok{family =} \KeywordTok{poisson}\NormalTok{(}\DataTypeTok{link=}\StringTok{'log'}\NormalTok{), }\DataTypeTok{data =}\NormalTok{ data_mil)}
\NormalTok{r2 <-}\StringTok{ }\KeywordTok{glm}\NormalTok{(}\DataTypeTok{formula =}\NormalTok{ formula_}\DecValTok{2}\NormalTok{, }\DataTypeTok{family =} \KeywordTok{poisson}\NormalTok{(}\DataTypeTok{link=}\StringTok{'log'}\NormalTok{), }\DataTypeTok{data =}\NormalTok{ data_mil)}
\NormalTok{r3 <-}\StringTok{ }\KeywordTok{glm}\NormalTok{(}\DataTypeTok{formula =}\NormalTok{ formula_}\DecValTok{3}\NormalTok{, }\DataTypeTok{family =} \KeywordTok{poisson}\NormalTok{(}\DataTypeTok{link=}\StringTok{'log'}\NormalTok{), }\DataTypeTok{data =}\NormalTok{ data_mil)}
\NormalTok{r4 <-}\StringTok{ }\KeywordTok{glm}\NormalTok{(}\DataTypeTok{formula =}\NormalTok{ formula_}\DecValTok{4}\NormalTok{, }\DataTypeTok{family =} \KeywordTok{poisson}\NormalTok{(}\DataTypeTok{link=}\StringTok{'log'}\NormalTok{), }\DataTypeTok{data =}\NormalTok{ data_mil)}

\NormalTok{se_rob <-}\StringTok{ }\KeywordTok{list}\NormalTok{(}\KeywordTok{sqrt}\NormalTok{(}\KeywordTok{diag}\NormalTok{(sandwich}\OperatorTok{::}\KeywordTok{vcovHC.default}\NormalTok{(r1,}\DataTypeTok{type =} \StringTok{"HC0"}\NormalTok{))),}
               \KeywordTok{sqrt}\NormalTok{(}\KeywordTok{diag}\NormalTok{(sandwich}\OperatorTok{::}\KeywordTok{vcovHC.default}\NormalTok{(r2,}\DataTypeTok{type =} \StringTok{"HC0"}\NormalTok{))),}
               \KeywordTok{sqrt}\NormalTok{(}\KeywordTok{diag}\NormalTok{(sandwich}\OperatorTok{::}\KeywordTok{vcovHC.default}\NormalTok{(r3,}\DataTypeTok{type =} \StringTok{"HC0"}\NormalTok{))),}
               \KeywordTok{sqrt}\NormalTok{(}\KeywordTok{diag}\NormalTok{(sandwich}\OperatorTok{::}\KeywordTok{vcovHC.default}\NormalTok{(r4,}\DataTypeTok{type =} \StringTok{"HC0"}\NormalTok{))))}
\end{Highlighting}
\end{Shaded}

\begin{Shaded}
\begin{Highlighting}[]
\NormalTok{stargazer}\OperatorTok{::}\KeywordTok{stargazer}\NormalTok{(r1,r2,r3,r4, }\DataTypeTok{title =} \StringTok{"Table 1 - Poisson regression"}\NormalTok{, }
          \DataTypeTok{type=}\KeywordTok{ifelse}\NormalTok{(knitr}\OperatorTok{::}\KeywordTok{is_latex_output}\NormalTok{(),}\StringTok{"latex"}\NormalTok{,}\StringTok{"html"}\NormalTok{), }\DataTypeTok{se =}\NormalTok{ se_rob, }\DataTypeTok{out =} \StringTok{"./images/RB_T_I.tex"}\NormalTok{)}
\end{Highlighting}
\end{Shaded}

\% Table created by stargazer v.5.2.2 by Marek Hlavac, Harvard
University. E-mail: hlavac at fas.harvard.edu \% Date and time: Mon, Oct
07, 2019 - 19:30:03

\begin{table}[!htbp] \centering 
  \caption{Table 1 - Poisson regression} 
  \label{tab:} 
\begin{tabular}{@{\extracolsep{5pt}}lcccc} 
\\[-1.8ex]\hline 
\hline \\[-1.8ex] 
 & \multicolumn{4}{c}{\textit{Dependent variable:}} \\ 
\cline{2-5} 
\\[-1.8ex] & \multicolumn{4}{c}{numbil0} \\ 
\\[-1.8ex] & (1) & (2) & (3) & (4)\\ 
\hline \\[-1.8ex] 
 lngdppc & 1.084$^{***}$ & 0.717$^{***}$ & 0.737$^{***}$ & 0.963$^{***}$ \\ 
  & (0.138) & (0.244) & (0.233) & (0.243) \\ 
  & & & & \\ 
 lnpop & 1.171$^{***}$ & 0.806$^{***}$ & 0.929$^{***}$ & 1.153$^{***}$ \\ 
  & (0.097) & (0.213) & (0.195) & (0.293) \\ 
  & & & & \\ 
 gattwto08 & 0.006 & 0.007 & 0.004 & 0.0003 \\ 
  & (0.007) & (0.006) & (0.006) & (0.004) \\ 
  & & & & \\ 
 lnmcap08 &  & 0.399$^{**}$ & 0.286$^{*}$ & 0.114 \\ 
  &  & (0.172) & (0.167) & (0.237) \\ 
  & & & & \\ 
 rintr &  & $-$0.010 & $-$0.009 & $-$0.007 \\ 
  &  & (0.010) & (0.010) & (0.009) \\ 
  & & & & \\ 
 topint08 &  & $-$0.051$^{***}$ & $-$0.058$^{***}$ & $-$0.060$^{***}$ \\ 
  &  & (0.011) & (0.012) & (0.015) \\ 
  & & & & \\ 
 nrrents &  &  & $-$0.005 & 0.013 \\ 
  &  &  & (0.010) & (0.013) \\ 
  & & & & \\ 
 roflaw &  &  & 0.203 & 0.342 \\ 
  &  &  & (0.372) & (0.283) \\ 
  & & & & \\ 
 fullprivproc &  &  &  & $-$0.002$^{*}$ \\ 
  &  &  &  & (0.001) \\ 
  & & & & \\ 
 Constant & $-$29.050$^{***}$ & $-$19.444$^{***}$ & $-$20.858$^{***}$ & $-$25.951$^{***}$ \\ 
  & (2.578) & (4.820) & (4.255) & (6.240) \\ 
  & & & & \\ 
\hline \\[-1.8ex] 
Observations & 197 & 131 & 131 & 113 \\ 
Log Likelihood & $-$438.540 & $-$259.731 & $-$256.024 & $-$179.661 \\ 
Akaike Inf. Crit. & 885.079 & 533.461 & 530.049 & 379.322 \\ 
\hline 
\hline \\[-1.8ex] 
\textit{Note:}  & \multicolumn{4}{r}{$^{*}$p$<$0.1; $^{**}$p$<$0.05; $^{***}$p$<$0.01} \\ 
\end{tabular} 
\end{table}

\section{Example II - Structural
estimation}\label{example-ii---structural-estimation}

In this section the aim is to estimate the parameters from the
likelihood function of a given model and be able to calculate it in the
statistical software (in this case, R). We are going to estimate the
structural parameters of a very simple search model
\citep{flinn1982new}, following
\href{https://pdfs.semanticscholar.org/d55e/6da87f3d2e328987b44fd5462114adda6ec6.pdf}{Flinn
and Heckman 1982}. This tutorial is not devoted to understanding such
labor model, so we are going to describe the model and its assumptions;
from that point we are going to derive the ML function. Then we are
going to feed that function to the computer as in the previous case and
maximize it to find the parameters of the model.

\subsubsection{The model:}\label{the-model}

Agents are well behaved and maximize the income they receive. When they
are unemployed, they face a searching cost \(c\). Upon paying such cost,
offers from a \textbf{Poisson process} will arrive. The arrival offer
rate is denoted by \(\lambda\), and its probability by unit of time is
also \(\lambda\). When they meet a \textbf{wage} is proposed from a
\textbf{log-normal distribution} and the individuals can refuse to form
the match or `seal the deal'. We also assume that the \emph{length of
the employment spells} follow a \textbf{exponential distribution} and
that there is a \textbf{constant risk of loosing the job} with period
probability \(\eta\). Time is discounted at a rate \(\rho\).

Such model is characterized by two Bellman equations. The first equation
is the \textbf{value of being employed} {[}eq. (3.2) of the paper{]}:

\[ V_e(w) =  \frac{1}{\rho}[w + (1-\eta)V_e + \eta V_u]\]

Reorganizing terms boils down to:

\[V_e(w) = \frac{1}{\rho + \eta}[w + \eta V_u ]\]

The second equation is the \textbf{value of being unemployed}:

\[V_u = \frac{1}{\rho}[c+(1-\lambda)V_u+ \lambda + \mathbf{E} \max \lbrace V_e, V_u \rbrace]\]

Which, after rearranging, is equal to:

\[\rho V_u = -c + \frac{\lambda}{\rho + \eta} \int_{\rho V_u}(w-\rho V_u)f(w)\]

These two equations describe the whole behaviour of the economy under
the assumptions of the model. Now let's list all the assumptions that we
have until now, since they are going to be usefull for the construction
of the likelihood function:

\begin{itemize}
\item
  The minimum accepted wage is equal to \(w^* = \rho V_u\). In the paper
  the minimum accepted wage is not estimated, but instead is taken from
  the data.
\item
  Arrival rate of offers: \(\lambda\).
\item
  Termination rate: \(\eta\)
\item
  Distribution of employment duration: \(f[t_e]= \eta e^{(-\eta t_e)}\);
  the average duration is then \(\mathbf{E}[t_e]= \eta^{-1}\).
\item
  Rate of leaving the unemployed is equal to: \(\lambda (1- F(w^*))\).
  We are going to call this \(h_u\)
\item
  Distribution of unemployment duration is exponential
  \(f[t_u]= h_u e^{-h_u t_u)}\). The expected value in unemployment then
  is then equal to \((\lambda (1- F(w^*)))^{-1}=h_u^{-1}\)
\item
  The distribution of accepted wages is equal to:
  \(\frac{f(w)}{1 - F(w^*)}\). The term below is to adjust it to a
  distribution since a proper distribution must integrate to 1 and our
  is left censored.
\item
  The distribution of employments spells is right-censored. Given that
  it is negative exponential it coincides with the population.
\end{itemize}

Considering all this information, we can proceed to calculate the
probability to sample an employed individual out of the population as
weel as the probability to sample an unemployed individual. Let's start
with the latter:

\[P(U)=\frac{\mathbf{E}[t_u]}{\mathbf{E}[t_u]+\mathbf{E}[t_e]}=\frac{h_u^{-1}}{h_u^{-1}+\eta^{-1}}=\frac{\eta}{h_u+\eta}\]
Now let's derive the probability of sampling an employed individual:

\[P(E)=\frac{\mathbf{E}[t_e]}{\mathbf{E}[t_u]+\mathbf{E}[t_e]}=\frac{\eta^{-1}}{h_u^{-1}+\eta^{-1}}=\frac{h_u}{h_u+\eta}\]

Usually, data on the duration of unemployment \(t_u^o\) and the wages
\(w^o\) are observed from census or surveys, hence we can calculate such
parameters because now we can define the likelihood:

\begin{equation}
\begin{aligned}
\mathcal{L}&=\prod_U [P(U) \times f(t_u^o)]\times \prod_E [P(E) \times f(w^o)]=\\
&= \prod_U\left[ P(U) \times \lambda (1- F(w^*)) e^{- \lambda (1- F(w^*)) t_u} \right] \times \prod_E \left[P(E) \times \frac{f(w|\mu, \sigma)}{1 - F(w^*)} \right]=\\
&=  \prod_U\left[\frac{\eta}{h_u+\eta}  \times \lambda (1- F(w^*)) e^{- \lambda (1- F(w^*)) t_u} \right] \times \prod_E \left[\frac{h_u}{h_u+\eta} \times \frac{f(w|\mu, \sigma)}{1 - F(w^*)} \right]=\\
&=  \prod_U\left[\frac{\eta}{\lambda (1- F(w^*))+\eta}  \times \lambda (1- F(w^*)) e^{- \lambda (1- F(w^*)) t_u} \right] \times\\
&\prod_E \left[\frac{\lambda (1- F(w^*))}{\lambda (1- F(w^*))+\eta} \times \frac{f(w|\mu, \sigma)}{1 - F(w^*)} \right]\\
\end{aligned}
\end{equation}

After taking logs and rearranging we obtain the following
\textbf{log-likelihood}:

\[\log \mathcal{L} = N \log(h_u) - h_u \sum t_u + N_u \log(\eta) + \sum f(w|\mu,\sigma) -  N_e \log(1-F(w^*))  - N_e \log(h_u + \eta)\]

We just need to feed this function to the optimizer with some data and
we can obtain the parameters of the model. Before carrying out the
liklihood estimation, we have a look at the data and try to parse out
some useful extra information.

\subsubsection{The Data}\label{the-data}

To estimate the model we just need two vectors of data: duration of
unemployment and hourly wages. To make a real example we are going to
use the Current population survey (CSP), but most of the household
surveys contain these kind of data (eg, colombian GEIH).

First we go to the \href{http://www.census.gov/cps/}{page of the CPS},
and learn about the nature of the data. We can also download the
historical monthly data from the
\href{https://data.nber.org/data/cps_basic.html}{NBER webpage}. For this
exercise we are going to use the January 2019 data, which can be
obtained following
\href{https://data.nber.org/cps-basic/jan19pub.zip}{this link}. Download
and unzip the dataset in your working directory. Take also a moment to
check the required variables, looking the
\href{https://data.nber.org/cps-basic/January_2017_Record_Layout.txt}{documentation
for the data}. The open the dataset using the functionality of the
\texttt{reader} package \citep{R-readr}.

\begin{Shaded}
\begin{Highlighting}[]
\NormalTok{data <-}\StringTok{ }\KeywordTok{read_csv}\NormalTok{(}\StringTok{"jan19pub.dat"}\NormalTok{, }\DataTypeTok{col_names =} \OtherTok{FALSE}\NormalTok{, }\DataTypeTok{trim_ws =} \OtherTok{FALSE}\NormalTok{)}
\end{Highlighting}
\end{Shaded}

As you see, the data only contain 1 vector of text -- in jargon,
observations are in long format. Each row is a long string of text. If
you took the time to read the documentation you will see that you can
find next to each variable the description and the location of the
specific information in such long string. To extract information about
unemployment duration and employment wages we need the following parts:

\begin{longtable}[]{@{}lllll@{}}
\toprule
GROUP OF INTEREST & VARIABLE & LENGTH & DESCRIPTION &
LOCATION\tabularnewline
\midrule
\endhead
EMPLOYED & HRMIS & 2 & MONTH-IN-SAMPLE & 63 - 64\tabularnewline
EMPLOYED & PEERNPER & 2 & PERIODICITY & 502 - 503\tabularnewline
EMPLOYED & PEERNHRO & 2 & USUAL HOURS & 525 - 526\tabularnewline
EMPLOYED & PRERNWA & 8 & WEEKLY EARNINGS RECODE & 527 -
534\tabularnewline
EMPLOYED & PRERNHLY & 4 & RECODE FOR HOURLY RATE & 520 -
523\tabularnewline
EMPLOYED & PEHRUSL1 & 2 & HOW MANY HOURS PER WEEK DO YOU WORK & 218 -
219\tabularnewline
EMPLOYED & PTWK & 1 & WEEKLY EARNINGS - TOP CODE & 535 -
535\tabularnewline
UNEMPLOYED & PEMLR & 2 & MONTHLY LABOR FORCE RECODE & 180 -
181\tabularnewline
UNEMPLOYED & RUNEDUR & 3 & DURATION OF UNEMPLOYMENT FOR & 407 -
409\tabularnewline
ALL & GESTFIPS & 2 & FEDERAL INFORMATION STATE & 93 - 94\tabularnewline
\bottomrule
\end{longtable}

After identifying this information we can filter the data. To do so we
create a database for the employed and a database for the unemployed,
since the identification requires different information and variables.
Let's begin with the employed: we create a \emph{`data.frame'}
containing the relevant colums and we extract the information by the
position in the string.

\begin{Shaded}
\begin{Highlighting}[]
\NormalTok{employed <-}\StringTok{ }\KeywordTok{data.frame}\NormalTok{(}\KeywordTok{matrix}\NormalTok{(}\OtherTok{NA}\NormalTok{, }\DataTypeTok{nrow =} \KeywordTok{nrow}\NormalTok{(data), }\DataTypeTok{ncol =} \DecValTok{8}\NormalTok{))}

\KeywordTok{colnames}\NormalTok{(employed) <-}\StringTok{ }\KeywordTok{c}\NormalTok{(  }\StringTok{"HRMIS"}\NormalTok{,}
                          \StringTok{"PEERNPER"}\NormalTok{,}
                          \StringTok{"PEERNHRO"}\NormalTok{,}
                          \StringTok{"PRERNWA"}\NormalTok{,}
                          \StringTok{"PRERNHLY"}\NormalTok{,}
                          \StringTok{"PTWK"}\NormalTok{,}
                          \StringTok{"PEHRUSL1"}\NormalTok{,}
                          \StringTok{"GESTFIPS"}\NormalTok{)}

\NormalTok{employed}\OperatorTok{$}\NormalTok{HRMIS          <-}\StringTok{ }\KeywordTok{apply}\NormalTok{(data,}\DecValTok{1}\NormalTok{, }\ControlFlowTok{function}\NormalTok{(x) }\KeywordTok{as.numeric}\NormalTok{(}\KeywordTok{substr}\NormalTok{(}\KeywordTok{toString}\NormalTok{(x),}\DecValTok{63}\NormalTok{,}\DecValTok{64}\NormalTok{)))}
\NormalTok{employed}\OperatorTok{$}\NormalTok{PEERNPER       <-}\StringTok{ }\KeywordTok{apply}\NormalTok{(data,}\DecValTok{1}\NormalTok{, }\ControlFlowTok{function}\NormalTok{(x) }\KeywordTok{as.numeric}\NormalTok{(}\KeywordTok{substr}\NormalTok{(}\KeywordTok{toString}\NormalTok{(x),}\DecValTok{502}\NormalTok{,}\DecValTok{503}\NormalTok{)))}
\NormalTok{employed}\OperatorTok{$}\NormalTok{PEERNHRO       <-}\StringTok{ }\KeywordTok{apply}\NormalTok{(data,}\DecValTok{1}\NormalTok{, }\ControlFlowTok{function}\NormalTok{(x) }\KeywordTok{as.numeric}\NormalTok{(}\KeywordTok{substr}\NormalTok{(}\KeywordTok{toString}\NormalTok{(x),}\DecValTok{525}\NormalTok{,}\DecValTok{526}\NormalTok{)))}
\NormalTok{employed}\OperatorTok{$}\NormalTok{PRERNWA        <-}\StringTok{ }\KeywordTok{apply}\NormalTok{(data,}\DecValTok{1}\NormalTok{, }\ControlFlowTok{function}\NormalTok{(x) }\KeywordTok{as.numeric}\NormalTok{(}\KeywordTok{substr}\NormalTok{(}\KeywordTok{toString}\NormalTok{(x),}\DecValTok{527}\NormalTok{,}\DecValTok{534}\NormalTok{)))}
\NormalTok{employed}\OperatorTok{$}\NormalTok{PRERNHLY       <-}\StringTok{ }\KeywordTok{apply}\NormalTok{(data,}\DecValTok{1}\NormalTok{, }\ControlFlowTok{function}\NormalTok{(x) }\KeywordTok{as.numeric}\NormalTok{(}\KeywordTok{substr}\NormalTok{(}\KeywordTok{toString}\NormalTok{(x),}\DecValTok{520}\NormalTok{,}\DecValTok{523}\NormalTok{)))}
\NormalTok{employed}\OperatorTok{$}\NormalTok{PTWK           <-}\StringTok{ }\KeywordTok{apply}\NormalTok{(data,}\DecValTok{1}\NormalTok{, }\ControlFlowTok{function}\NormalTok{(x) }\KeywordTok{as.numeric}\NormalTok{(}\KeywordTok{substr}\NormalTok{(}\KeywordTok{toString}\NormalTok{(x),}\DecValTok{535}\NormalTok{,}\DecValTok{535}\NormalTok{)))}
\NormalTok{employed}\OperatorTok{$}\NormalTok{PEHRUSL1       <-}\StringTok{ }\KeywordTok{apply}\NormalTok{(data,}\DecValTok{1}\NormalTok{, }\ControlFlowTok{function}\NormalTok{(x) }\KeywordTok{as.numeric}\NormalTok{(}\KeywordTok{substr}\NormalTok{(}\KeywordTok{toString}\NormalTok{(x),}\DecValTok{218}\NormalTok{,}\DecValTok{219}\NormalTok{)))}
\NormalTok{employed}\OperatorTok{$}\NormalTok{GESTFIPS       <-}\StringTok{ }\KeywordTok{apply}\NormalTok{(data,}\DecValTok{1}\NormalTok{, }\ControlFlowTok{function}\NormalTok{(x) }\KeywordTok{substr}\NormalTok{(}\KeywordTok{toString}\NormalTok{(x),}\DecValTok{93}\NormalTok{,}\DecValTok{94}\NormalTok{))}
\end{Highlighting}
\end{Shaded}

Reading the documentation we identify that the invalid cases are coded
\(0\) or negative values \(-1,-2,\dots\). We reclassify this coded
infromation as \texttt{NA}, a \emph{not available} or \emph{missing
value}. We are going to keep the observations that are from the outgoing
rotations (have earnings information). After that we are going to filter
the valid hourly wages and convert the weekkly wages to hours using the
number of hours. We are going to keep only the people that have this
information. We are also going to take the right number of decimal for
the variables that specify it. We are going to keep only the
observations that are above the legal federal minimum wage of \(7.5\)
USD, and we are going to trim the data at the percentile 99.9\%.

\begin{Shaded}
\begin{Highlighting}[]
\NormalTok{employed[employed}\OperatorTok{<=}\DecValTok{0}\NormalTok{] <-}\StringTok{ }\OtherTok{NA}
        
\NormalTok{employed <-}\StringTok{ }\NormalTok{employed[}\KeywordTok{which}\NormalTok{(employed}\OperatorTok{$}\NormalTok{HRMIS }\OperatorTok\StringTok{ }\KeywordTok{c}\NormalTok{(}\DecValTok{4}\NormalTok{,}\DecValTok{8}\NormalTok{)),]}
\NormalTok{employed <-}\StringTok{ }\NormalTok{employed[}\KeywordTok{which}\NormalTok{(employed}\OperatorTok{$}\NormalTok{PEERNPER }\OperatorTok{>}\StringTok{ }\DecValTok{0}\NormalTok{),]}
\NormalTok{employed <-}\StringTok{ }\NormalTok{employed[}\OperatorTok{-}\KeywordTok{which}\NormalTok{(employed}\OperatorTok{$}\NormalTok{PRERNHLY }\OperatorTok{==}\StringTok{ }\DecValTok{9999}\NormalTok{),]}
\NormalTok{employed}\OperatorTok{$}\NormalTok{PRERNHLY <-}\StringTok{ }\NormalTok{employed}\OperatorTok{$}\NormalTok{PRERNHLY}\OperatorTok{/}\DecValTok{100}
\NormalTok{employed <-}\StringTok{ }\NormalTok{employed[}\OperatorTok{-}\KeywordTok{which}\NormalTok{(employed}\OperatorTok{$}\NormalTok{PTWK }\OperatorTok{==}\StringTok{ }\DecValTok{1}\NormalTok{),]}
\NormalTok{employed}\OperatorTok{$}\NormalTok{PRERNWA <-}\StringTok{ }\NormalTok{employed}\OperatorTok{$}\NormalTok{PRERNWA}\OperatorTok{/}\DecValTok{100}

\NormalTok{employed}\OperatorTok{$}\NormalTok{wages <-}\StringTok{ }\KeywordTok{ifelse}\NormalTok{(employed}\OperatorTok{$}\NormalTok{PRERNHLY }\OperatorTok{>}\StringTok{ }\DecValTok{0} \OperatorTok{&}\StringTok{ }\OperatorTok{!}\KeywordTok{is.na}\NormalTok{(employed}\OperatorTok{$}\NormalTok{PRERNHLY),}
\NormalTok{                         employed}\OperatorTok{$}\NormalTok{PRERNHLY,}
                         \KeywordTok{ifelse}\NormalTok{(}\OperatorTok{!}\KeywordTok{is.na}\NormalTok{(employed}\OperatorTok{$}\NormalTok{PRERNWA) }\OperatorTok{&}\StringTok{ }\OperatorTok{!}\KeywordTok{is.na}\NormalTok{(employed}\OperatorTok{$}\NormalTok{PEHRUSL1),}
\NormalTok{                                employed}\OperatorTok{$}\NormalTok{PRERNWA}\OperatorTok{/}\NormalTok{employed}\OperatorTok{$}\NormalTok{PEHRUSL1,}
                                \OtherTok{NA}\NormalTok{)}
\NormalTok{                         )}

\NormalTok{employed <-}\StringTok{ }\NormalTok{employed[}\KeywordTok{which}\NormalTok{(}\OperatorTok{!}\KeywordTok{is.na}\NormalTok{(employed}\OperatorTok{$}\NormalTok{wages)),]}
\NormalTok{employed <-}\StringTok{ }\NormalTok{employed[}\KeywordTok{which}\NormalTok{(employed}\OperatorTok{$}\NormalTok{wages }\OperatorTok{>=}\StringTok{ }\FloatTok{7.25}\NormalTok{),]}
\NormalTok{employed <-}\StringTok{ }\NormalTok{employed[}\KeywordTok{which}\NormalTok{(employed}\OperatorTok{$}\NormalTok{wages }\OperatorTok{<=}\StringTok{ }\KeywordTok{quantile}\NormalTok{(employed}\OperatorTok{$}\NormalTok{wages, }\FloatTok{0.999}\NormalTok{)),]}

\NormalTok{tokeep_e <-}\StringTok{ }\KeywordTok{c}\NormalTok{(}\StringTok{"GESTFIPS"}\NormalTok{, }\StringTok{"wages"}\NormalTok{)}
\NormalTok{employed <-}\StringTok{ }\NormalTok{employed[,tokeep_e]}
\NormalTok{employed}\OperatorTok{$}\NormalTok{duration_U <-}\StringTok{ }\DecValTok{0}
\end{Highlighting}
\end{Shaded}

Now we apply the same selection to the unemployed: We collect the
infromation in each variable following the position. We take the people
for which the labor employment status is unemployment and that have
information on the duration. Then we convert the information to monthly
data.

\begin{Shaded}
\begin{Highlighting}[]
\NormalTok{unemployed <-}\StringTok{ }\KeywordTok{data.frame}\NormalTok{(}\KeywordTok{matrix}\NormalTok{(}\OtherTok{NA}\NormalTok{, }\DataTypeTok{nrow =} \KeywordTok{nrow}\NormalTok{(data), }\DataTypeTok{ncol =} \DecValTok{3}\NormalTok{))}
\KeywordTok{colnames}\NormalTok{(unemployed) <-}\StringTok{ }\KeywordTok{c}\NormalTok{(}\StringTok{"PEMLR"}\NormalTok{,}
                        \StringTok{"RUNEDUR"}\NormalTok{,}
                        \StringTok{"GESTFIPS"}\NormalTok{)}

\NormalTok{unemployed}\OperatorTok{$}\NormalTok{PEMLR          <-}\StringTok{ }\KeywordTok{apply}\NormalTok{(data,}\DecValTok{1}\NormalTok{, }\ControlFlowTok{function}\NormalTok{(x) }\KeywordTok{as.numeric}\NormalTok{(}\KeywordTok{substr}\NormalTok{(}\KeywordTok{toString}\NormalTok{(x),}\DecValTok{180}\NormalTok{,}\DecValTok{181}\NormalTok{)))}
\NormalTok{unemployed}\OperatorTok{$}\NormalTok{RUNEDUR        <-}\StringTok{ }\KeywordTok{apply}\NormalTok{(data,}\DecValTok{1}\NormalTok{, }\ControlFlowTok{function}\NormalTok{(x) }\KeywordTok{as.numeric}\NormalTok{(}\KeywordTok{substr}\NormalTok{(}\KeywordTok{toString}\NormalTok{(x),}\DecValTok{407}\NormalTok{,}\DecValTok{409}\NormalTok{)))}
\NormalTok{unemployed}\OperatorTok{$}\NormalTok{GESTFIPS       <-}\StringTok{ }\KeywordTok{apply}\NormalTok{(data,}\DecValTok{1}\NormalTok{, }\ControlFlowTok{function}\NormalTok{(x) }\KeywordTok{substr}\NormalTok{(}\KeywordTok{toString}\NormalTok{(x),}\DecValTok{93}\NormalTok{,}\DecValTok{94}\NormalTok{))}

\NormalTok{unemployed <-}\StringTok{ }\NormalTok{unemployed[}\KeywordTok{which}\NormalTok{(unemployed}\OperatorTok{$}\NormalTok{PEMLR }\OperatorTok\StringTok{ }\KeywordTok{c}\NormalTok{(}\DecValTok{3}\NormalTok{,}\DecValTok{4}\NormalTok{)),]}
\NormalTok{unemployed <-}\StringTok{ }\NormalTok{unemployed[}\KeywordTok{which}\NormalTok{(unemployed}\OperatorTok{$}\NormalTok{RUNEDUR }\OperatorTok{>}\StringTok{ }\DecValTok{0}\NormalTok{),]}
\NormalTok{unemployed}\OperatorTok{$}\NormalTok{duration_U <-}\StringTok{ }\NormalTok{unemployed}\OperatorTok{$}\NormalTok{RUNEDUR}\OperatorTok{/}\FloatTok{4.333}
\NormalTok{unemployed <-}\StringTok{ }\NormalTok{unemployed[,}\KeywordTok{c}\NormalTok{(}\StringTok{"GESTFIPS"}\NormalTok{,}\StringTok{"duration_U"}\NormalTok{)]}
\NormalTok{unemployed}\OperatorTok{$}\NormalTok{wages <-}\StringTok{ }\DecValTok{0} 
\NormalTok{unemployed <-}\StringTok{ }\NormalTok{unemployed[,}\KeywordTok{c}\NormalTok{(}\StringTok{"GESTFIPS"}\NormalTok{,}\StringTok{"wages"}\NormalTok{, }\StringTok{"duration_U"}\NormalTok{)]}
\end{Highlighting}
\end{Shaded}

As a final step, we merge the two dataset:

\begin{Shaded}
\begin{Highlighting}[]
\NormalTok{data <-}\StringTok{ }\KeywordTok{rbind}\NormalTok{(employed,unemployed)}
\end{Highlighting}
\end{Shaded}

We are going to select and calculate the MLE using only one state, as
minimum wages laws vary locally. For this exercise we are going to use
the information on Nevada (\(GESTFIPS = 32\))

\begin{Shaded}
\begin{Highlighting}[]
\NormalTok{data_sub <-}\StringTok{ }\NormalTok{data[}\KeywordTok{which}\NormalTok{(data}\OperatorTok{$}\NormalTok{GESTFIPS}\OperatorTok{==}\StringTok{"32"}\NormalTok{),]}
\end{Highlighting}
\end{Shaded}

\subsubsection{Estimation}\label{estimation-1}

In order to estimate the log likelihood we are going to code two
auxiliary funcions that appear all the time in the procedure. The first
one is the
\href{https://en.wikipedia.org/wiki/Log-normal_distribution}{log-normal
density function}:

\[LN \left(x; \mu, \sigma\right) = \phi \left( \frac{\log(x) - \mu}{\sigma} \right)(\frac{1}{x \sigma})\]
Where \(\phi\) is the probability density function of the \(N(0,1)\)
distribution.

\begin{Shaded}
\begin{Highlighting}[]
\NormalTok{lognorm <-}\StringTok{ }\ControlFlowTok{function}\NormalTok{(x,mu,sigma)\{}
\NormalTok{        res <-}\StringTok{ }\KeywordTok{dnorm}\NormalTok{((}\KeywordTok{log}\NormalTok{(x)}\OperatorTok{-}\NormalTok{mu)}\OperatorTok{/}\NormalTok{(sigma))}\OperatorTok{*}\NormalTok{(}\DecValTok{1}\OperatorTok{/}\NormalTok{(sigma}\OperatorTok{*}\NormalTok{x))}
        \KeywordTok{return}\NormalTok{(res)}
\NormalTok{\}}
\end{Highlighting}
\end{Shaded}

The second function is the \emph{survival function}, which is the
probability to be over the minimum accepted wage for a given
distribution. We define the survival as \((1- F(w^*))\). The cumulative
distribution function of the log-normal is equal to:

\[\Phi \left( \frac{\log(x) - \mu}{\sigma} \right)\] Where \(\Phi\) is
the cumulative distribution function of the standard normal
distribution.

\begin{Shaded}
\begin{Highlighting}[]
\NormalTok{surv <-}\StringTok{ }\ControlFlowTok{function}\NormalTok{(val,mu,sigma)\{}
\NormalTok{        res <-}\StringTok{ }\DecValTok{1} \OperatorTok{-}\StringTok{ }\KeywordTok{pnorm}\NormalTok{((}\KeywordTok{log}\NormalTok{(val)}\OperatorTok{-}\NormalTok{mu)}\OperatorTok{/}\NormalTok{(sigma))}
        \KeywordTok{return}\NormalTok{(res)}
\NormalTok{\}}
\end{Highlighting}
\end{Shaded}

There are other built-in functions already developed that are suitable
for this kind of problems. One xample is the \texttt{mle2} function from
the \emph{`bbmle'} package \citep{R-bbmle}. This is useful since the
function deals with standard errors and provides other information that
might be useful. There are some difference between the \texttt{optim()}
method already covered and the \texttt{mle2()} function. This latter
function requires:

\begin{itemize}
\item
  Function to calculate negative log-likelihood
\item
  Starting values for the optimizer
\item
  The optimizer used
\item
  The data
\end{itemize}

Now we are going to code the log-likelihood function we recovered using
the same procedure as in the previous example:

First we are going to code each of the parameters and the data as inputs
in the function. Then we are going to code the likelihood using the
definition derived earlier.

\begin{Shaded}
\begin{Highlighting}[]
\NormalTok{LLfnct_mle <-}\StringTok{ }\ControlFlowTok{function}\NormalTok{(alambda,aeta,amu,asigma, data)\{}
        
        
\NormalTok{        lambda =}\StringTok{ }\KeywordTok{exp}\NormalTok{(alambda) }
\NormalTok{        eta    =}\StringTok{ }\KeywordTok{exp}\NormalTok{(aeta) }
\NormalTok{        mu     =}\StringTok{ }\NormalTok{amu}
\NormalTok{        sigma  =}\StringTok{ }\KeywordTok{exp}\NormalTok{(asigma) }
        
\NormalTok{        w_star <-}\StringTok{ }\KeywordTok{min}\NormalTok{(data[}\KeywordTok{which}\NormalTok{(data}\OperatorTok{$}\NormalTok{wages }\OperatorTok{>}\StringTok{ }\DecValTok{0}\NormalTok{),]}\OperatorTok{$}\NormalTok{wages)}
        
        
\NormalTok{        h_u <-}\StringTok{ }\NormalTok{lambda }\OperatorTok{*}\StringTok{ }\KeywordTok{surv}\NormalTok{(w_star, mu, sigma)}
\NormalTok{        n <-}\StringTok{ }\KeywordTok{nrow}\NormalTok{(data)}
\NormalTok{        n_u <-}\StringTok{  }\KeywordTok{nrow}\NormalTok{(data[}\KeywordTok{which}\NormalTok{(data}\OperatorTok{$}\NormalTok{duration_U }\OperatorTok{>}\StringTok{ }\DecValTok{0}\NormalTok{),])}
\NormalTok{        n_e <-}\StringTok{ }\NormalTok{(n}\OperatorTok{-}\NormalTok{n_u)}
        
\NormalTok{        LL <-}\StringTok{   }\NormalTok{n }\OperatorTok{*}\StringTok{ }\KeywordTok{log}\NormalTok{(h_u) }\OperatorTok{-}
\StringTok{                }\NormalTok{n_e }\OperatorTok{*}\StringTok{ }\KeywordTok{log}\NormalTok{(}\KeywordTok{surv}\NormalTok{(w_star, mu, sigma)) }\OperatorTok{-}
\StringTok{                }\NormalTok{h_u }\OperatorTok{*}\StringTok{ }\KeywordTok{sum}\NormalTok{(data}\OperatorTok{$}\NormalTok{duration_U) }\OperatorTok{+}
\StringTok{                }\NormalTok{n_u }\OperatorTok{*}\StringTok{ }\KeywordTok{log}\NormalTok{(eta) }\OperatorTok{+}
\StringTok{                }\KeywordTok{sum}\NormalTok{(}\KeywordTok{log}\NormalTok{(}\KeywordTok{lognorm}\NormalTok{(data[}\KeywordTok{which}\NormalTok{(data}\OperatorTok{$}\NormalTok{wages}\OperatorTok{>}\DecValTok{0}\NormalTok{),]}\OperatorTok{$}\NormalTok{wages,mu,sigma))) }\OperatorTok{-}
\StringTok{                }\NormalTok{n_e }\OperatorTok{*}\StringTok{ }\KeywordTok{log}\NormalTok{(eta }\OperatorTok{+}\StringTok{ }\NormalTok{h_u)}
        
        \CommentTok{#print(cbind(c("lambda = ","eta = ","mu = ","sigma = "),c(lambda,eta,mu,sigma)))}
        \KeywordTok{return}\NormalTok{(}\OperatorTok{-}\NormalTok{LL)}
        
\NormalTok{\}}
\end{Highlighting}
\end{Shaded}

After that we just have to set up the maximizer and feed it the
function:

\begin{Shaded}
\begin{Highlighting}[]
\NormalTok{m0 <-}\StringTok{ }\KeywordTok{mle2}\NormalTok{(LLfnct_mle, }
           \DataTypeTok{start =} \KeywordTok{list}\NormalTok{(}\DataTypeTok{alambda =}\OperatorTok{-}\DecValTok{1}\NormalTok{,}
                        \DataTypeTok{aeta =} \OperatorTok{-}\DecValTok{1}\NormalTok{, }
                        \DataTypeTok{amu =} \DecValTok{3}\NormalTok{,}
                        \DataTypeTok{asigma =} \OperatorTok{-}\DecValTok{1}\NormalTok{), }
           \DataTypeTok{data =} \KeywordTok{list}\NormalTok{(}\DataTypeTok{data =}\NormalTok{ data_sub),}
           \DataTypeTok{optimizer =} \StringTok{"nlminb"}\NormalTok{)}

\NormalTok{m0}\OperatorTok{@}\NormalTok{coef}
\end{Highlighting}
\end{Shaded}

\begin{verbatim}
##    alambda       aeta        amu     asigma 
## -0.4541099 -1.8175680  2.7146674 -0.4361533
\end{verbatim}

All the coeficients need to be transformed to recover and present the
results. Before that we are going to calculate the standard errors using
the delta method and the information from the hessian. At the end what
we are doing is just rescaling the errors.

\begin{Shaded}
\begin{Highlighting}[]
\NormalTok{lambda <-}\StringTok{ }\KeywordTok{exp}\NormalTok{(m0}\OperatorTok{@}\NormalTok{coef[}\StringTok{"alambda"}\NormalTok{])}
\NormalTok{eta <-}\StringTok{ }\KeywordTok{exp}\NormalTok{(m0}\OperatorTok{@}\NormalTok{coef[}\StringTok{"aeta"}\NormalTok{])}
\NormalTok{mu <-}\StringTok{ }\NormalTok{m0}\OperatorTok{@}\NormalTok{coef[}\StringTok{"amu"}\NormalTok{]}
\NormalTok{sigma <-}\StringTok{ }\KeywordTok{exp}\NormalTok{(m0}\OperatorTok{@}\NormalTok{coef[}\StringTok{"asigma"}\NormalTok{])}


\NormalTok{fisher_info <-}\StringTok{ }\NormalTok{m0}\OperatorTok{@}\NormalTok{details}\OperatorTok{$}\NormalTok{hessian}
\NormalTok{vcov_mle <-}\StringTok{ }\KeywordTok{solve}\NormalTok{(fisher_info)}
\NormalTok{prop_sigma<-}\KeywordTok{sqrt}\NormalTok{(}\KeywordTok{c}\NormalTok{(lambda}\OperatorTok{^}\DecValTok{2}\NormalTok{,eta}\OperatorTok{^}\DecValTok{2}\NormalTok{,mu}\OperatorTok{^}\DecValTok{2}\NormalTok{,sigma}\OperatorTok{^}\DecValTok{2}\NormalTok{) }\OperatorTok{*}\StringTok{ }\KeywordTok{diag}\NormalTok{(vcov_mle))}
\KeywordTok{names}\NormalTok{(prop_sigma) <-}\StringTok{ }\KeywordTok{c}\NormalTok{(}\StringTok{"lambda"}\NormalTok{,}\StringTok{"eta"}\NormalTok{,}\StringTok{"mu"}\NormalTok{,}\StringTok{"sigma"}\NormalTok{)}
\end{Highlighting}
\end{Shaded}

\begin{Shaded}
\begin{Highlighting}[]
\KeywordTok{stargazer}\NormalTok{(}\KeywordTok{cbind}\NormalTok{(}\StringTok{"parameter"}\NormalTok{ =}\StringTok{ }\KeywordTok{names}\NormalTok{(prop_sigma), }\StringTok{"theta"}\NormalTok{ =}\KeywordTok{c}\NormalTok{(lambda,eta,mu,sigma), prop_sigma),}
          \DataTypeTok{title =} \StringTok{"Model parameters MLE - Nevada (Jan 2019)"}\NormalTok{, }
          \DataTypeTok{type=}\KeywordTok{ifelse}\NormalTok{(knitr}\OperatorTok{::}\KeywordTok{is_latex_output}\NormalTok{(),}\StringTok{"latex"}\NormalTok{,}\StringTok{"html"}\NormalTok{), }\DataTypeTok{out =} \StringTok{"./images/SE_Nevada.tex"}\NormalTok{)}
\end{Highlighting}
\end{Shaded}

\% Table created by stargazer v.5.2.2 by Marek Hlavac, Harvard
University. E-mail: hlavac at fas.harvard.edu \% Date and time: Mon, Oct
07, 2019 - 19:30:37

\begin{table}[!htbp] \centering 
  \caption{Model parameters MLE - Nevada (Jan 2019)} 
  \label{tab:} 
\begin{tabular}{@{\extracolsep{5pt}} cccc} 
\\[-1.8ex]\hline 
\hline \\[-1.8ex] 
 & parameter & theta & prop\_sigma \\ 
\hline \\[-1.8ex] 
alambda & lambda & 0.635012931441372 & 0.0900528154150777 \\ 
aeta & eta & 0.16242027479781 & 0.0379917018868457 \\ 
amu & mu & 2.71466743713899 & 0.270516421768113 \\ 
asigma & sigma & 0.646518617780906 & 0.0678137201168286 \\ 
\hline \\[-1.8ex] 
\end{tabular} 
\end{table}

\section{References}\label{references-1}

This material was possible thanks to the slides of David MARGOLIS (in
PSE resources), the course of C. FLinn at CCA 2017, and
\href{https://python.quantecon.org/mle.html}{QuantEcon MLE}.

\bibliography{book.bib,packages.bib}


\end{document}
